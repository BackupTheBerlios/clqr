% Copyright (C) 2008 Bert Burgemeister
%
% Permission is granted to copy, distribute and/or modify this
% document under the terms of the GNU Free Documentation License,
% Version 1.2 or any later version published by the Free Software
% Foundation; with no Invariant Sections, no Front-Cover Texts and
% no Back-Cover Texts. For details see file COPYING.
%

%%%%%%%%%%%%%%%%%%%%%%%%%%%%%%%%%%%%%%%%%%%%%%%%%%
\section{Arrays} 
%%%%%%%%%%%%%%%%%%%%%%%%%%%%%%%%%%%%%%%%%%%%%%%%%%
\label{section:Arrays}

%%%%%%%%%%%%%%%%%%%%%%%%%%%%%%%%%%%%%%%%%%%%%%%%%%
\subsection{Predicates} 
%%%%%%%%%%%%%%%%%%%%%%%%%%%%%%%%%%%%%%%%%%%%%%%%%%
\begin{LIST}{1cm}

  \IT{\arrGOO{%
      (\FU*{ARRAYP} \VAR{ foo})\\
      (\FU*{VECTORP} \VAR{ foo})\\
      (\FU*{SIMPLE-VECTOR-P} \VAR{ foo})\\
      (\FU*{BIT-VECTOR-P} \VAR{ foo})\\
      (\FU*{SIMPLE-BIT-VECTOR-P} \VAR{ foo})}{\}}}
  {
  \retval{\T} if \VAR{foo} is of indicated type.
  }

  \IT{\arrGOO{(\FU*{ADJUSTABLE-ARRAY-P} \VAR{ array})\\
      (\FU*{ARRAY-HAS-FILL-POINTER-P} \VAR{ array})}{.}}
  {
  Return \retval{\T} if \VAR{array} is adjustable/has a fill pointer,
  respectively. 
  }

  \IT{(\FU*{ARRAY-IN-BOUNDS-P} \VAR{array} \Op{\VAR{subscripts}})}
  {
  Return \retval{\T} if \VAR{subscripts} are in \VAR{array}'s bounds.
  }

\end{LIST}


%%%%%%%%%%%%%%%%%%%%%%%%%%%%%%%%%%%%%%%%%%%%%%%%%%
\subsection{Array Functions} 
%%%%%%%%%%%%%%%%%%%%%%%%%%%%%%%%%%%%%%%%%%%%%%%%%%
\begin{LIST}{1cm}

  \IT{(\xorGOO{\FU*{MAKE-ARRAY}\\
      \FU*{ADJUST-ARRAY} \VAR{ array}}{\}}
    \VAR{dimensions}
    \orGOO{\kwd{:element-type} \VAR{ type}\DF{\T}\\
      \kwd{:adjustable} \VAR{ bool}\DF{\NIL}\\
      \kwd{:fill-pointer } \Goo{\VAR{num}\XOR\VAR{bool}}\DF{\NIL}\\
      \xorGOO{\kwd{:initial-element} \VAR{ obj}\\
        \kwd{:initial-contents} \VAR{ seq}\\
        \kwd{:displaced-to} \xorGOO{\VAR{array}\\\VAR{bool}\DF{\NIL}}{\}}
        \Op{\kwd{:displaced-index-offset } \VAR{i}\DF{0}}}{.}}{\}})}
  { 
  Return fresh, or readjust, respectively, \retval{vector or array of
    dimension(s) \VAR{dim}}. 
  }

  \IT{(\kwd*{AREF} \VAR{array} \OP{\VAR{subscripts}})}
  {
  Return
  \retval{array element} pointed to by \VAR{subscripts}. \kwd{setf}able.
  }

  \IT{(\FU*{ROW-MAJOR-AREF} \VAR{array} \VAR{i})}
  {
  Return \retval{\VAR{i}th element} of \VAR{array} in row-major order.
  }

  \IT{(\FU*{ARRAY-ROW-MAJOR-INDEX} \VAR{array} \VAR{subscripts})}
  {
  \retval{Index} in row-major order of element denoted by \VAR{subscripts}.
  }

  \IT{(\FU*{ARRAY-DIMENSIONS} \VAR{array})}
  {
  \retval{List} containing \VAR{array}'s dimensions. 
  }

  \IT{(\FU*{ARRAY-DIMENSION} \VAR{array} \VAR{i})}
  {
  \retval{length of \VAR{i}th dimension} of \VAR{array}.
  }

  \IT{(\FU*{ARRAY-TOTAL-SIZE} \VAR{array})}
  {
  \retval{Number of elements} in \VAR{array}.
  }

  \IT{(\FU*{ARRAY-DISPLACEMENT} \VAR{array})}
  {
  \retval{Target array} and \retval{offset}.
  }

   \IT{(\FU*{ARRAY-RANK} \VAR{array})}
   {
  \retval{Number of dimensions} of \VAR{array}.
  }

  \IT{\arrGOO{(\FU*{BIT} \VAR{ bit-array} \VAR{ subscripts})\\
      (\FU*{SBIT} \VAR{ simple-bit-array} \VAR{ subscripts})}{.}}
  {
  Return \retval{element} of \VAR{bit-array} or of
  \VAR{simple-bit-array}. \kwd{setf}able. 
  }

  \IT{(\FU*{BIT-NOT} \VAR{bit-array} \Op{\VAR{arg}\DF{\NIL}})}
  {
  Return \retval{result} of bit-wise negation of \VAR{bit-array}. If
  \VAR{arg} is a bit array, put result there. If \VAR{arg} is \T, put
  result in \VAR{bit-array}. If \VAR{arg} is \NIL, make a new array for
  result. 
  }

  \IT{(\xorGOO{\FU*{BIT-AND}\\
      \FU*{BIT-ANDC1}\\
      \FU*{BIT-ANDC2}\\
      \FU*{BIT-EQV}\\
      \FU*{BIT-IOR}\\
      \FU*{BIT-NAND}\\
      \FU*{BIT-NOR}\\
      \FU*{BIT-ORC1}\\
      \FU*{BIT-ORC2}\\
      \FU*{BIT-XOR}}{\}} \VAR{bit-array-a} \VAR{bit-array-b}
    \Op{\VAR{arg}\DF{\NIL}})}
  {
  Return \retval{result} of bit-wise logical operations on
  \VAR{bit-array-a} and  \VAR{bit-array-b}. If \VAR{arg} is a bit
  array, put result there. If \VAR{arg} is \T, put result in
  \VAR{bit-array-a}. If \VAR{arg} is \NIL, make a new array for
  result. 
  }

  \IT{\CNS*{ARRAY-RANK-LIMIT}}
  {
  Upper bound of array rank, $\geq 8$.
  }

  \IT{\CNS*{ARRAY-DIMENSION-LIMIT}}
  {
  Upper bound of an array dimension, $\geq 1024$.
  }

  \IT{\CNS*{ARRAY-TOTAL-SIZE-LIMIT}}
  {
  Upper bound of array size, $\geq 1024$.
  }

\end{LIST}


%%%%%%%%%%%%%%%%%%%%%%%%%%%%%%%%%%%%%%%%%%%%%%%%%%
\subsection{Vector Functions} 
%%%%%%%%%%%%%%%%%%%%%%%%%%%%%%%%%%%%%%%%%%%%%%%%%%

Vectors can as well be manipulated by sequence functions, see
s.\ \ref{section:Sequences}. 

\begin{LIST}{1cm}
  
  \IT{(\FU*{VECTOR} \OPn{\VAR{foo}})}
  {
  Return fresh \retval{vector of \VAR{foo}s}.
  }

  \IT{(\kwd*{SVREF} \VAR{vector} \VAR{i})}
  {
  Return \retval{\VAR{i}th
    element} of \VAR{vector}. \kwd{setf}able.
  }

  \IT{(\FU*{VECTOR-PUSH} \VAR{foo} \VAR{vector})}
  {
  Return \retval{\NIL} if \VAR{vector}'s fill pointer equals size of
  \VAR{vector}. Otherwise replace element of \VAR{vector} pointed to
  by \retval{fill pointer} with \VAR{foo}, then increment fill
  pointer. 
  }

  \IT{(\FU*{VECTOR-PUSH-EXTEND} \VAR{foo} \VAR{vector}
    \Op{\VAR{num}})}
  {
  Replace element of \VAR{vector} pointed to by \retval{fill pointer} with
  \VAR{foo}, then increment fill pointer. Extend \VAR{vector}'s size by
  $\ge \VAR{num}$ if necessary.
  }

  \IT{(\FU*{VECTOR-POP} \VAR{vector})}
  {
  Return \retval{element of \VAR{vector}} its fillpointer points to
  after decrementation.
  }

  \IT{(\FU*{FILL-POINTER} \VAR{vector})}
  {
  Return \retval{fill pointer} of \VAR{vector}. \kwd{setf}able.
  }

\end{LIST}



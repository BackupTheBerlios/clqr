% Copyright (C) 2008 Bert Burgemeister
%
% Permission is granted to copy, distribute and/or modify this
% document under the terms of the GNU Free Documentation License,
% Version 1.2 or any later version published by the Free Software
% Foundation; with no Invariant Sections, no Front-Cover Texts and
% no Back-Cover Texts. For details see file COPYING.
%

%%%%%%%%%%%%%%%%%%%%%%%%%%%%%%%%%%%%%%%%%%%%%%%%%%
\section{Strings} 
%%%%%%%%%%%%%%%%%%%%%%%%%%%%%%%%%%%%%%%%%%%%%%%%%%
Strings can as well be manipulated by array and sequence functions,
see pages \pageref{section:Arrays} and \pageref{section:Sequences}.

\begin{LIST}{1cm}


  \IT{\arrGOO{(\FU*{STRINGP} \VAR{ foo})\\
      (\FU*{SIMPLE-STRING-P} \VAR{ foo})}{.}}
  {
  Return \retval{\T} if \VAR{foo} is of type
  \kwd{string} or \kwd{simple-string}, respectively.
  }

  \IT{(\xorGOO{\FU*{STRING=}\\\FU*{STRING-EQUAL}}{\}} \VAR{foo}
    \VAR{bar} 
    \orGOO{\kwd{:start1} \VAR{ start-foo}\DF{\LIT{0}}\\
      \kwd{:start2} \VAR{ start-bar}\DF{\LIT{0}}\\
      \kwd{:end1} \VAR{ end-foo}\DF{\NIL}\\
      \kwd{:end2} \VAR{ end-bar}\DF{\NIL}}{\}})}
  {
  Return \retval{\T} if
  subsequences of \VAR{foo} and \VAR{bar} are equal. Obey/ignore,
  respectively, case. 
  }

  \IT{(\xorGOO{\FU*{STRING/=}\\\FU{STRING\boldmath$>$}\\
      \FU{STRING\boldmath$>=$}\\\FU{STRING\boldmath$<$}\\
      \FU{STRING\boldmath$<=$}}{\}} \VAR{foo} \VAR{bar}
    \orGOO{\kwd{:start1} \VAR{ start-foo}\DF{\LIT{0}}\\
      \kwd{:start2} \VAR{ start-bar}\DF{\LIT{0}}\\
      \kwd{:end1} \VAR{ end-foo}\DF{\NIL}\\
      \kwd{:end2} \VAR{ end-bar}\DF{\NIL}}{\}})}
  {
  \index{string>@STRING$>$}%
  \index{string>=@STRING$>=$}%
  \index{string<@STRING$<$}%
  \index{string<=@STRING$<=$}%
  If \VAR{foo} is
  lexicographically not equal, greater, not less, less, or not greater,
  respectively, then return \retval{character number} from beginning of \VAR{foo}
  where they begin to differ.  Otherwise return \retval{\NIL}.
  }

  \IT{(\xorGOO{\FU*{STRING-NOT-EQUAL}\\
      \FU*{STRING-GREATERP}\\
      \FU*{STRING-NOT-LESSP}\\
      \FU*{STRING-LESSP}\\
      \FU*{STRING-NOT-GREATERP}}{\}} 
    \VAR{foo} \VAR{bar}
    \orGOO{%
      \kwd{:start1} \VAR{ start-foo}\DF{\LIT{0}}\\
      \kwd{:start2} \VAR{ start-bar}\DF{\LIT{0}}\\
      \kwd{:end1} \VAR{ end-foo}\DF{\NIL}\\
      \kwd{:end2} \VAR{ end-bar}\DF{\NIL}}{\}})}
  {
  If \VAR{foo} is
  lexicographically not equal, greater, not less, less, or not greater,
  respectively, ignoring case, then return \retval{character number} from beginning of \VAR{foo}
  where they begin to differ. Otherwise return \retval{\NIL}.
  }

  \IT{(\FU*{STRING} \VAR{x})}
  {
  Convert \VAR{x} (\kwd{symbol}, \kwd{string}, or \kwd{character})
  into a \retval{string}. 
  }

  \IT{(\FU*{MAKE-STRING} \VAR{size} \orGOO{\kwd{:initial-element} \VAR{ char}\\
      \kwd{:element-type} \VAR{ type}\DF{\kwd{character}}}{\}})}
  {
  Return \retval{string} of length \VAR{size}.
  }

  \IT{(\xorGOO{\FU{STRING}\\
      \FU{NSTRING}}{\}}\kwd{-}\xorGOO{\kwd{CAPITALIZE}\\
      \kwd{UPCASE}\\
      \kwd{DOWNCASE}}{\}}
    \VAR{string}  
    \orGOO{\kwd{:start} \VAR{ start}\DF{\LIT{0}}\\
      \kwd{:end} \VAR{ end}\DF{\NIL}}{\}})}
  {
  \index{STRING-CAPITALIZE}%
  \index{STRING-UPCASE}%
  \index{STRING-DOWNCASE}%
  \index{NSTRING-CAPITALIZE}%
  \index{NSTRING-UPCASE}%
  \index{NSTRING-DOWNCASE}%
  Return \retval{\VAR{string}} (not modified or modified, respectively) with
  first letter of every word turned into uppercase, letters all
  uppercase, or letters all lowercase, respectively. 
  }

  \IT{(\xorGOO{\FU*{STRING-TRIM}\\
      \FU*{STRING-LEFT-TRIM}\\
      \FU*{STRING-RIGHT-TRIM}}{\}} \VAR{char-bag} \VAR{string})}
  {
  Return \retval{\VAR{string}} with all characters in sequence \VAR{char-bag} removed
  from both ends, from the beginning, or from the end, respectively.
  }

  \IT{\arrGOO{(\FU*{CHAR} \VAR{ string} \VAR{ i})\\
      (\FU*{SCHAR} \VAR{ string} \VAR{ i})}{.}}
  {
  Return zero-indexed \retval{\VAR{i}th character} of string ignoring/obeying,
  respectively, fill pointer. \kwd{setf}able. 
  }

  \IT{(\FU*{PARSE-INTEGER} \VAR{string}
    \orGOO{\kwd{:start} \VAR{ start}\DF{\LIT{0}}\\
      \kwd{:end} \VAR{ end}\DF{\NIL}\\
      \kwd{:radix} \VAR{ int}\DF{10}\\
      \kwd{:junk-allowed} \VAR{ bool}\DF{\NIL}}{\}})}
  {
  Parse \retval{integer} from \VAR{string}. Second value is
  \retvalii{index} of parse end in \VAR{string}.
  }



\end{LIST}

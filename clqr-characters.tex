% Copyright (C) 2008, 2010 Bert Burgemeister
%
% Permission is granted to copy, distribute and/or modify this
% document under the terms of the GNU Free Documentation License,
% Version 1.2 or any later version published by the Free Software
% Foundation; with no Invariant Sections, no Front-Cover Texts and
% no Back-Cover Texts. For details see file COPYING.
%

%%%%%%%%%%%%%%%%%%%%%%%%%%%%%%%%%%%%%%%%%%%%%%%%%%
\section{Characters} 
%%%%%%%%%%%%%%%%%%%%%%%%%%%%%%%%%%%%%%%%%%%%%%%%%%

The \kwd*{standard-char} type comprises \LIT{a}-\LIT{z},
\LIT{A}-\LIT{Z}, \LIT{0}-\LIT{9}, \LIT{Newline}, \LIT{Space}, and
\LIT{!?\$"'`.:,;*+-/|{\tt\char`\\}\TLD\_\^{}<=>\#\%@\&()[]\{\}}.
\index{NEWLINE}\index{SPACE}

\begin{LIST}{1cm}

  \IT{\arrGOO{(\FU*{CHARACTERP} \VAR{ foo})\\
      (\FU*{STANDARD-CHAR-P} \VAR{ char})}{.}}
  {
    \retval{\T} if argument is of indicated type.
  }

  \IT{\arrGOO{(\FU*{GRAPHIC-CHAR-P} \VAR{ character})\\
      (\FU*{ALPHA-CHAR-P} \VAR{ character})\\
      (\FU*{ALPHANUMERICP} \VAR{ character})}{.}}
  {
    \retval{\T} if \VAR{character} is visible, alphabetic, or
    alphanumeric, respectively.
  }

  \IT{\arrGOO{(\FU*{UPPER-CASE-P} \VAR{ character})\\
      (\FU*{LOWER-CASE-P} \VAR{ character})\\
      (\FU*{BOTH-CASE-P} \VAR{ character})}{.}}
  {
  Return \retval{\T} if \VAR{character} is uppercase, lowercase, or
  able to be in another case, respectively.
  }

  \IT{(\FU*{DIGIT-CHAR-P} \VAR{character}
    \Op{\VAR{radix}\DF{\LIT{10}}})}
  {
    Return \retval{its weight} if \VAR{character} is a digit, or
    \retval{\NIL} otherwise.
  }

  \IT{\arrGOO{(\FU*{CHAR=}\RP{\VAR{
          character}})\\(\FU*{CHAR/=}\RP{\VAR{
          character}})}{.}}
  {
    Return \retval{\T} if all \VAR{character}s, or
    none, respectively,  are equal.
  }

  \IT{\arrGOO{(\FU*{CHAR-EQUAL}\RP{\VAR{
          character}})\\(\FU*{CHAR-NOT-EQUAL}\RP{\VAR{
          character}})}{.}}
  {
    Return \retval{\T} if all \VAR{character}s, or
    none, respectively,  are equal ignoring case.
  }

  \IT{\arrGOO{(\FU{CHAR\boldmath$>$}\RP{\VAR{
          character}})\\(\FU{CHAR\boldmath$>$=}\RP{\VAR{
          character}})\\(\FU{CHAR\boldmath$<$}\RP{\VAR{
          character}})\\(\FU{CHAR\boldmath$<$=}\RP{\VAR{
          character}})}{.}}
  {
  \index{CHAR>@CHAR$>$}%
  \index{CHAR>=@CHAR$>$=}%
  \index{CHAR<@CHAR$<$}%
  \index{CHAR<=@CHAR$<$=}%
  Return \retval{\T} if \VAR{character}s are
  monotonically decreasing, monotonically non-increasing, monotonically
  increasing, or monotonically non-decreasing, respectively.
  }

  \IT{\arrGOO{
      (\FU*{CHAR-GREATERP}\RP{\VAR{ character}})\\ 
      (\FU*{CHAR-NOT-LESSP}\RP{\VAR{ character}})\\ 
      (\FU*{CHAR-LESSP}\RP{\VAR{ character}})\\ 
      (\FU*{CHAR-NOT-GREATERP}\RP{\VAR{ character}})}{.}}
  {
    Return \retval{\T} if \VAR{character}s are monotonically
    decreasing, monotonically non-increasing, monotonically
    increasing, or monotonically non-decreasing, respectively,
    ignoring case.
  }

  \IT{\arrGOO{(\FU*{CHAR-UPCASE} \VAR{ character})\\
      (\FU*{CHAR-DOWNCASE} \VAR{ character})}{.}}
  {
    Return corresponding uppercase/lowercase \retval{character},
    respectively.
  }

  \IT{(\FU*{DIGIT-CHAR} \VAR{i} \Op{\VAR{radix}\DF{\LIT{10}}})}
  {
    \retval{Character} representing digit \VAR{i}.
  }

  \IT{(\FU*{CHAR-NAME} \VAR{character})}
  {
    \VAR{character}'s \retval{name} if any, or
    \retval{\NIL}. 
  }

  \IT{(\FU*{NAME-CHAR} \VAR{foo})}
  {
    \retval{Character} named \VAR{foo} if any, or
    \retval{\NIL}.
  }

  \IT{\arrGOO{(\FU*{CHAR-INT} \VAR{ character})\\
      (\FU*{CHAR-CODE} \VAR{ character})}{.}}
  {
    \retval{Code} of \VAR{character}.
  }

  \IT{(\FU*{CODE-CHAR} \VAR{code})}
  {
    \retval{Character} with \VAR{code}.
  }

  \IT{\CNS*{CHAR-CODE-LIMIT}}
  {
    Upper bound of (\FU{CHAR-CODE} \VAR{char}); $\geq 96$.
  }

  \IT{(\FU*{CHARACTER} \VAR{c})}
  {
    Return \retval{\kwd{\#$\backslash$}\VAR{c}}.
  }

\end{LIST}

%%% Local Variables: 
%%% mode: latex
%%% TeX-master: "clqr"
%%% End: 

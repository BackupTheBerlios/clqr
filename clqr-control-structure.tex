% Copyright (C) 2008 Bert Burgemeister
%
% Permission is granted to copy, distribute and/or modify this
% document under the terms of the GNU Free Documentation License,
% Version 1.2 or any later version published by the Free Software
% Foundation; with no Invariant Sections, no Front-Cover Texts and
% no Back-Cover Texts. For details see file COPYING.
%

%%%%%%%%%%%%%%%%%%%%%%%%%%%%%%%%%%%%%%%%%%%%%%%%%%
\section{Control Structure}
%%%%%%%%%%%%%%%%%%%%%%%%%%%%%%%%%%%%%%%%%%%%%%%%%%

%%%%%%%%%%%%%%%%%%%%%%%%%%%%%%%%%%%%%%%%%%%%%%%%%%
\subsection{Predicates}
%%%%%%%%%%%%%%%%%%%%%%%%%%%%%%%%%%%%%%%%%%%%%%%%%%
\begin{LIST}{1cm}

  \IT{(\FU*{EQ} \VAR{foo bar})}
  {
  \retval{\T} if \VAR{foo} and \VAR{bar} are identical.
  }

  \IT{(\FU*{EQL} \VAR{foo bar})}
  {
  \retval{\T} if \VAR{foo} and \VAR{bar}
  are identical, or the same \kwd{character}, or \kwd{number}s of the
  same type and value. 
  }

  \IT{(\FU*{EQUAL} \VAR{foo bar})}
  {
  \retval{\T} if \VAR{foo} and \VAR{bar}
  are \FU{EQL}, or are equivalent \kwd{pathname}s, or are \kwd{cons}es with
  \FU{equal} cars and cdrs, or are \kwd{string}s or \kwd{bit-vector}s with
  \FU{eql} elements below their fill pointers.
  }

  \IT{(\FU*{EQUALP} \VAR{foo bar})}
  {
  \retval{\T} if \VAR{foo} and \VAR{bar} are identical; or are the
  same \kwd{character} ignoring case; or are \kwd{number}s of the same
  value ignoring type; or are equivalant \kwd{pathname}s; or are
  \kwd{cons}es or \kwd{array}s of the same shape with \FU{equalp}
  elements; or are structures of the same \kwd{class} with
  \kwd{equalp} elements; or are \kwd{hash-table}s of the same size
  with the same \kwd{:test} function, the same keys in terms of
  \kwd{:test} function, and \kwd{equalp} elements.
  }

  \IT{(\FU*{NOT} \VAR{foo})}
  {
  \retval{\T} if \VAR{foo} is \NIL, \retval{\NIL} otherwise.
  }

  \IT{(\FU*{BOUNDP} \VAR{symbol})}
  {
    \retval{\T} if \VAR{symbol} is a special variable.
  }

  \IT{(\FU*{CONSTANTP} \VAR{foo} \Op{\VAR{environment}\DF{\NIL}})}
  {
  \retval{\T} if \VAR{foo} is a constant form.
  }

  \IT{(\FU*{FUNCTIONP} \VAR{foo})}
  {
  \retval{\T} if \VAR{foo} is of type \kwd{function}.
  }

  \IT{(\FU*{FBOUNDP} \xorGOO{\VAR{foo}\\
      (\kwd{setf } \VAR{foo})}{\}})}
  {
  \retval{\T} if \VAR{foo} is a global function or macro.
  }


\end{LIST}


%%%%%%%%%%%%%%%%%%%%%%%%%%%%%%%%%%%%%%%%%%%%%%%%%%
\subsection{Variables}
%%%%%%%%%%%%%%%%%%%%%%%%%%%%%%%%%%%%%%%%%%%%%%%%%%

\begin{LIST}{1cm}

  \IT{(\xorGOO{%
      \MC*{DEFCONSTANT}\\
      \MC*{DEFPARAMETER}}{\}} \NEV{\VAR{foo}} \VAR{form}
      \Op{\NEV{\VAR{doc}}})}
  {
  Assign value of \VAR{form} to global constant/dynamic variable \retval{\VAR{foo}}.
  }

  \IT{(\MC*{DEFVAR} \NEV{\VAR{foo}} \OP{\VAR{form}
      \Op{\NEV{\VAR{doc}}}})}
  {
  Unless bound already, assign value of \VAR{form} to dynamic variable
  \retval{\VAR{foo}}. 
  }

  \IT{(\xorGOO{\MC*{SETF}\\
      \MC*{PSETF}}{\}} \Goos{\VAR{place}
      \VAR{form}})}
  {
  Set \VAR{place}s to primary values of \VAR{form}s. Return \retval{values
    of last \VAR{form}}/\retval{\NIL}; work sequentially/in parallel, respectively. 
  }

  \IT{(\xorGOO{\SO*{SETQ}\\
      \MC*{PSETQ}}{\}} \Goos{\VAR{symbol}
      \VAR{form}})}
  {
  Set \VAR{symbol}s to primary values of \VAR{form}s. Return \retval{value
    of last \VAR{form}}/\retval{\NIL}; work sequentially/in parallel, respectively. 
  }

  \IT{(\FU*{SET} \DES{\VAR{symbol}} \VAR{foo})}
  {
  Set \VAR{symbol}'s value cell to \retval{\VAR{foo}}. Deprecated.
  }

  \IT{(\MC*{MULTIPLE-VALUE-SETQ} \VAR{vars} \VAR{form})}
  {
  Set elements of \VAR{vars} to the values of
  \VAR{form}. Return \retval{\VAR{form}'s primary value}.
  }

  \IT{(\MC*{SHIFTF} \RP{\DES{\VAR{place}}} \VAR{foo})}
  {
  Store value of \VAR{foo} in rightmost \VAR{place} shifting values of
  \VAR{place}s left, returning \retval{first \VAR{place}}. 
  }

  \IT{(\MC*{ROTATEF} \OPn{\DES{\VAR{place}}})}
  {
  Rotate values of \VAR{place}s left, old first becoming new last
  \VAR{place}'s value. Return \retval{\NIL}.
  }

  \IT{(\FU*{MAKUNBOUND} \DES{\VAR{foo}})}
  {
  Delete special variable \retval{\VAR{foo}} if any.
  }

  \label{:property_lists}
  \IT{\arrGOO{(\FU*{GET} \VAR{ symbol} \VAR{ key }
      \OP{\VAR{default}\DF{\NIL}})\\ 
      (\FU*{GETF} \VAR{ place} \VAR{ key }
      \OP{\VAR{default}\DF{\NIL}})}{.}}
  {
  \retval{First entry \VAR{key}} from property list stored in
  \VAR{symbol}/in \VAR{place}, respectively, or \retval{\VAR{default}} if
  there is no \VAR{key}. \kwd{setf}able.
  }

  \IT{(\FU*{GET-PROPERTIES} \VAR{property-list} \VAR{keys})}
  {
  Return \retval{key} and \retvalii{value} of first entry from \VAR{property-list}
  matching a key from \VAR{keys}, and \retvaliii{tail of
    \VAR{property-list}} starting with that key. Return \retval{\NIL},
  \retvalii{\NIL}, and  \retvaliii{\NIL} if there was no matching key
  in \VAR{property-list}.
  }

  \IT{\arrGOO{(\FU*{REMPROP } \DES{\VAR{symbol}} \VAR{ key})\\
      (\MC*{REMF } \DES{\VAR{place}} \VAR{ key})}{.}}
  {
  Remove first entry \VAR{key} from property list stored in
  \VAR{symbol}/in \VAR{place}, respectively. Return \retval{\T} if \VAR{key}
  was there, or \retval{\NIL} otherwise.
  }

\end{LIST}


%%%%%%%%%%%%%%%%%%%%%%%%%%%%%%%%%%%%%%%%%%%%%%%%%%
\subsection{Functions}
%%%%%%%%%%%%%%%%%%%%%%%%%%%%%%%%%%%%%%%%%%%%%%%%%%
\label{section:Functions}

\begin{flushleft}
  Below, ordinary lambda list (\OPn{\VAR{ord-$\lambda$}}) has the form\\
  % \VAR{var} referenced from DEFSTRUCT
  (\OPn{\VAR{var}}
  \OP{\kwd*{\&optional} \xorGOO{%
      \VAR{var}\\
      (\VAR{var } \OP{\VAR{init}\DF{\NIL}\text{ } \Op{\VAR{supplied-p}}})}{\}}}
  \penalty-5
  \Op{\kwd*{\&rest} \VAR{var}}
  \penalty-5
  \OP{\kwd*{\&key} \xorGOO{%
        \VAR{var}\\
        (\xorGOO{%
          \VAR{var}\\
          (\kwd{:}\VAR{key } \VAR{var})}{\}}
      \text{ }\OP{\VAR{init}\DF{\NIL}\text{ }\Op{\VAR{supplied-p}}})}{\}^{\!\!*}}
  \penalty-5
    \Op{\kwd*{\&allow-other-keys}}}
  \penalty-5
  \OP{\kwd*{\&aux} \xorGOO{%
      \VAR{var}\\
      (\VAR{var } \Op{\VAR{init}\DF{\NIL}})}{\}}}).\\
  \VAR{supplied-p} is \T\ if there is a corresponding
  argument. \VAR{init} forms can refer to any \VAR{init} and
  \VAR{supplied-p} to their left.
\end{flushleft}

\begin{LIST}{1cm}

  \IT{(\xorGOO{\MC*{DEFUN }\xorGOO{\VAR{foo}\\
      (\kwd{setf} \VAR{ foo})}{.}\\ 
      \MC*{LAMBDA}}{\}}
    (\VAR{\OPn{ord-$\lambda$}}) \OPn{(\kwd{declare} \OPn{\NEV{\VAR{decl}}})}
    \Op{\NEV{\VAR{doc}}} 
    \PROGN{\VAR{form}})}
  {
  Define a function named \retval{\VAR{foo}} or \retval{(\kwd{setf}
  \VAR{foo})}, or an anonymous \retval{function}, respectively, which applies
  \VAR{form}s to \VAR{ord-$\lambda$}s. For \MC{defun}, \VAR{form}s are
  enclosed in an implicit \SO{block} \VAR{foo}.
  }

  \IT{(\xorGOO{\SO*{FLET}\\
      \SO*{LABELS}}{\}}
    (\OPn{(\xorGOO{\VAR{foo}\\
        (\kwd{setf} \VAR{ foo})}{\}}
      (\OPn{\VAR{ord-$\lambda$}}) 
      \OPn{(\kwd{declare} \OPn{\NEV{\VAR{local-decl}}})}
      \Op{\NEV{\VAR{doc}}}
      \PROGN{\VAR{local-form}})}) \OPn{(\kwd{declare} \OPn{\NEV{\VAR{decl}}})}
    \PROGN{\VAR{form}})}
  {
  Evaluate \VAR{form}s with
  locally defined functions \VAR{foo}. Each \VAR{foo} is also the name
  of an implicit \SO{block} around its corresponding
  \OPn{\VAR{local-form}}.  Only for \SO{LABELS}, functions \VAR{foo} are visible inside \VAR{local-forms}. Return
  \retval{values of \VAR{form}s}.
  }

  \IT{(\SO*{FUNCTION} \xorGOO{%
      \VAR{foo}\\
      (\MC{lambda } \OPn{\VAR{form}})}{\}})}
  {
    Return lexically innermost \retval{function} named \VAR{foo} or a
    lexical closure of the \retval{\MC{lambda} expression}.
  }

  \IT{(\FU*{APPLY} \xorGOO{\VAR{function}\\
      (\kwd{setf } \VAR{function})}{\}} \RP{\VAR{arg}})}
  {
  Return \retval{values of \VAR{function}} called on
  \VAR{arg}s. Last \VAR{arg} must be a list. \kwd{setf}able if
  \VAR{function} is one of \FU{aref}, \FU{bit}, and \FU{sbit}.
  }

  \IT{(\FU*{FUNCALL} \VAR{function} \OPn{arg})}
  {
  Return \retval{values of \VAR{function}} called with \VAR{arg}s.
  }

  \IT{(\SO*{MULTIPLE-VALUE-CALL} \VAR{foo} \OPn{\VAR{form}})}
  {Call function
  \VAR{foo} with all the values of each \VAR{form} as its arguments.  Return
  \retval{values returned by \VAR{foo}}.
  }

  \IT{(\FU*{VALUES-LIST} \VAR{list})}
  {
  Return \retval{elements of \VAR{list}}.
  }

  \IT{(\FU*{VALUES} \OPn{\VAR{foo}})}
  {
  Return as multiple values the \retval{primary values} of the
  \VAR{foo}s. \kwd{setf}able. 
  }

  \IT{(\FU*{MULTIPLE-VALUE-LIST} \VAR{form})}
  {
  Return in a \retval{list} values of \VAR{form}.
  }

  \IT{(\MC*{NTH-VALUE} \VAR{n} \VAR{form})}
  {
  Zero-indexed \retval{\VAR{n}th return value} of \VAR{form}.
  }

  \IT{(\FU*{COMPLEMENT} \VAR{function})}
  {
  Return \retval{new function} with same arguments and same side effects
  as \VAR{function}, but with complementary truth value.
  }

  \IT{(\FU*{CONSTANTLY} \VAR{foo})}
  {
  Return \retval{function} of any number of arguments returning \VAR{foo}.
  }

  \IT{(\FU*{IDENTITY} \VAR{foo})}
  {
  Return \retval{\VAR{foo}}.
  }

  \IT{(\FU*{FUNCTION-LAMBDA-EXPRESSION} \VAR{function})}
  {
  If available, return \retval{lambda expression} of \VAR{function},
  \retvalii{\NIL} if \VAR{function} was defined in an environment
  without bindings, and \retvaliii{name} of \VAR{function}. 
  }

  \IT{(\FU*{FDEFINITION} \xorGOO{\VAR{foo}\\
      (\kwd{setf } \VAR{foo})}{\}})}
  {
  \retval{Definition} of global function \VAR{foo}. \kwd{setf}able.
  }

  \IT{(\FU*{FMAKUNBOUND} \VAR{foo})}
  {
  Remove global function or macro definition \retval{\VAR{foo}}.
  }

  \IT{\arrGOO{\CNS*{CALL-ARGUMENTS-LIMIT}\\
      \CNS*{LAMBDA-PARAMETERS-LIMIT}}{.}}
  {
  Upper bound of the number of function arguments or lambda list
  parameters, respectively; $\geq50$. 
  }

  \IT{\CNS*{MULTIPLE-VALUES-LIMIT}}
  {
  Upper bound of the number of values a multiple value can have;
  $\geq20$. 
  }

\end{LIST}


%%%%%%%%%%%%%%%%%%%%%%%%%%%%%%%%%%%%%%%%%%%%%%%%%%
\subsection{Macros}
%%%%%%%%%%%%%%%%%%%%%%%%%%%%%%%%%%%%%%%%%%%%%%%%%%
\label{section:Macros}

\begin{flushleft}
  Below, macro lambda list (\OPn{\VAR{macro-$\lambda$}}) has the form of
  either
  \penalty-5
  (\Op{\kwd*{\&whole} \VAR{var}}
  \penalty-5
  \Op{\VAR{E}}
  \penalty-5
  \xorGOO{%
    \VAR{var}\\
    (\OPn{\VAR{macro-$\lambda$}})}{\}^{\!\!*}}
  \penalty-5
  \Op{\VAR{E}}
  \penalty-5
  \Op{\kwd*{\&optional} 
    \xorGOO{%
      \VAR{var}\\
      (\xorGOO{%
        \VAR{var}\\
        (\OPn{\VAR{macro-$\lambda$}})}{\}}\text{ }
      \OP{\VAR{init}\DF{\NIL}\text{ } \Op{\VAR{supplied-p}}})}{\}^{\!\!*}}}
  \penalty-5
  \Op{\VAR{E}}
  \penalty-5
  \Op{%
    \xorGOO{%
      \kwd*{\&rest}\\
      \kwd*{\&body}}{\}}
    \xorGOO{%
      \VAR{var}\\
      (\OPn{\VAR{macro-$\lambda$}})}{\}}}
  \penalty-5
  \Op{\VAR{E}}
  \penalty-5
  \OP{\kwd*{\&key} 
    \xorGOO{%
      \VAR{var}\\
      (\xorGOO{%
        \VAR{var}\\
        (\kwd{:}\VAR{key } \xorGOO{%
          \VAR{var}\\
          (\OPn{\VAR{macro-$\lambda$}})}{\}})}{\}}
      \text{ }\OP{\VAR{init}\DF{\NIL}\text{ } \Op{\VAR{supplied-p}}})}{\}^{\!\!*}}
    \Op{\VAR{E}}
    \Op{\kwd*{\&allow-other-keys}}}
  \penalty-5
  \Op{\kwd*{\&aux} 
    \xorGOO{%
      \VAR{var}\\
      (\VAR{var } \Op{\VAR{init}\DF{\NIL}})}{\}^{\!\!*}}}
  \penalty-5
  \Op{\VAR{E}})
  \\
  or
  \penalty-5
  (\Op{\kwd*{\&whole}
      \VAR{var}}
  \penalty-5
  \Op{\VAR{E}}
  \penalty-5
  \xorGOO{%
    \VAR{var}\\
    (\OPn{\VAR{macro-$\lambda$}})}{\}^{\!\!*}}
  \penalty-5
  \Op{\VAR{E}}
  \penalty-5
  \Op{\kwd*{\&optional} 
    \xorGOO{%
      \VAR{var}\\
      (\xorGOO{%
        \VAR{var}\\
        (\OPn{\VAR{macro-$\lambda$}})}{\}}\text{ }
      \OP{\VAR{init}\DF{\NIL}\text{ } \Op{\VAR{supplied-p}}})}{\}^{\!\!*}}}
  \penalty-5
  \Op{\VAR{E}}
  \kwd{.}
  \VAR{var}).
  \penalty-5
\end{flushleft}
  One toplevel \Op{\VAR{E}} may be replaced by \kwd*{\&environment}
  \VAR{var} where \VAR{var} carries the lexical compilation
  environment. \VAR{supplied-p} is \T\ if there is a corresponding
  argument. 

\begin{LIST}{1cm}

  \IT{(\xorGOO{\MC*{DEFMACRO}\\
      \FU*{DEFINE-COMPILER-MACRO}}{\}}
    \xorGOO{\VAR{foo}\\
    (\kwd{setf } \VAR{foo})}{\}}
    (\OPn{\VAR{macro-$\lambda$}})
    \OPn{(\kwd{declare} \OPn{\NEV{\VAR{decl}}})} \Op{\NEV{\VAR{doc}}}
    \PROGN{\VAR{form}})}
  {
  Define macro \retval{\VAR{foo}} which on evaluation as (\VAR{foo}
  \VAR{tree}) applies expanded \VAR{form}s to arguments from
  \VAR{tree} which corresponds to \VAR{tree}-shaped
  \VAR{macro-$\lambda$}s. \VAR{form}s are enclosed in an implicit
  \SO{block} \VAR{foo}.
  }

  \IT{(\MC*{DEFINE-SYMBOL-MACRO} \VAR{foo} \VAR{form})}
  {
  Define symbol macro \retval{\VAR{foo}} which on evaluation evaluates
  expanded \VAR{form}. 
  }

  \IT{(\SO*{MACROLET} (\OPn{(\VAR{foo} (\OPn{\VAR{macro-$\lambda$}})
      \OPn{(\kwd{declare} \OPn{\NEV{\VAR{local-decl}}})} 
      \Op{\NEV{\VAR{doc}}} \PROGN{\VAR{macro-form}})}) \OPn{(\kwd{declare}
      \OPn{\NEV{\VAR{decl}}})} \PROGN{\VAR{form}})}
  {
  Evaluate \retval{\VAR{form}s} with locally defined mutually
  invisible macros \VAR{foo} which are enclosed in implicit \SO{block}s
  of the same name.
  }

  \IT{(\SO*{SYMBOL-MACROLET} (\OPn{(\VAR{foo}
      \VAR{expansion-form})}) \OPn{(\kwd{declare} \OPn{\NEV{\VAR{decl}}})}
    \PROGN{\VAR{form}})}
  {
  Evaluate \retval{\VAR{form}s} with locally defined symbol macros
  \VAR{foo}. 
  }

  \IT{(\MC*{DEFSETF} \NEV{\VAR{function}} \xorGOO{%
      \NEV{\VAR{updater}} \text{ } \Op{\NEV{\VAR{doc}}}\\
      (\OPn{\VAR{setf-$\lambda$}}) \text{ } (\OPn{\VAR{s-var}}) \text{ } 
        \OPn{(\kwd{declare } \OPn{\NEV{\VAR{decl}}})}\text{ }
        \Op{\NEV{\VAR{doc}}} \text{ } \PROGN{\VAR{form}}}{\}})
    \penalty-5
  where defsetf lambda list (\OPn{\VAR{setf-$\lambda$}}) has the form
  \penalty-5
  (\OPn{\VAR{var}}
  \OP{\kwd*{\&optional} \xorGOO{%
      \VAR{var}\\
      (\VAR{var } \OP{\VAR{init}\DF{\NIL}\text{ } \Op{\VAR{supplied-p}}})}{\}}}
  \penalty-5
  \Op{\kwd*{\&rest} \VAR{var}}
  \penalty-5
  \OP{\kwd*{\&key} \xorGOO{%
      \VAR{var}\\
      (\xorGOO{%
        \VAR{var}\\
        (\kwd{:}\VAR{key } \VAR{var})}{\}}
      \text{ }\OP{\VAR{init}\DF{\NIL}\text{ }\Op{\VAR{supplied-p}}})}{\}^{\!\!*}}
    \penalty-5
    \Op{\kwd*{\&allow-other-keys}}}
  \penalty-5
  \OP{\kwd*{\&environment} \VAR{var}}%
  )
  }
  {
    Specify how to \kwd{setf} a place accessed by
    \retval{\VAR{function}}.
    Short form: (\kwd{setf} (\VAR{function} \OPn{\VAR{arg}}) \VAR{value-form}) is
    replaced by (\VAR{updater} \OPn{\VAR{arg}} \VAR{value-form}).
    Long form: on invocation of (\kwd{setf} (\VAR{function}
    \OPn{\VAR{arg}}) \VAR{value-form}), \VAR{form}s must expand
    into code that sets the place accessed where \VAR{setf-$\lambda$}
    and \OPn{\VAR{s-var}} describe the arguments of \VAR{function} and
    the value(s) to be stored, respectively; and that returns
    the value(s) of \OPn{\VAR{s-var}}. \VAR{form}s are enclosed in an
    implicit \SO{block} \VAR{foo}.
  }

  \IT{(\MC*{DEFINE-SETF-EXPANDER} \VAR{function}
    (\OPn{\VAR{macro-$\lambda$}})
    \OPn{(\kwd{declare} \OPn{\NEV{\VAR{decl}}})} \Op{\NEV{\VAR{doc}}}
    \PROGN{\VAR{form}})}
  {
  Specify how to \kwd{setf} a place accessed by
  \retval{\VAR{function}}. On invocation of (\kwd{setf} (\VAR{function}
    \OPn{\VAR{arg}}) \VAR{value-form}), \OPn{\VAR{form}} must expand
    into code returning \VAR{arg-vars}, \VAR{args}, \VAR{newval-vars},
    \VAR{set-form}, and \VAR{get-form} as described with
    \FU{GET-SETF-EXPANSION} where the elements of macro
    lambda list \OPn{\VAR{macro-$\lambda$}} are bound to corresponding
    \VAR{arg}s. \VAR{form}s are enclosed in an implicit \SO{block} \VAR{function}.
  }

  \IT{(\FU*{GET-SETF-EXPANSION} \VAR{place} \Op{\VAR{environment}\DF{\NIL}})}
  {
    Return lists of temporary variables \retval{\VAR{arg-vars}} and of
    corresponding \retvalii{\VAR{args}} as given with \VAR{place},
    list \retvaliii{\VAR{\VAR{newval-vars}}} with temporary variables
    corresponding to the new values, and \retvaln{4}{\VAR{set-form}}
    and \retvaln{5}{\VAR{get-form}} specifying in terms of \VAR{arg-vars}
    and \VAR{newval-vars} how to \kwd{setf} and how to read
    \VAR{place}. 
  }

  \IT{(\MC*{DEFINE-MODIFY-MACRO} \VAR{foo} 
    (\OP{\kwd*{\&optional} \xorGOO{%
        \VAR{var}\\
        (\VAR{var } \OP{\VAR{init}\DF{\NIL}\text{ }
          \Op{\VAR{supplied-p}}})}{\}}}
    \Op{\kwd*{\&rest} \VAR{var}}) 
    \VAR{function} \Op{\NEV{\VAR{doc}}})}
  {
  Define macro \retval{\VAR{foo}} able to modify a place. On
  invocation of (\VAR{foo} \VAR{place} \OPn{\VAR{arg}}), the value of
  \VAR{function} applied to \VAR{place} and \VAR{arg}s will be stored into
  \VAR{place} and returned.
  } 

  \IT{\CNS*{LAMBDA-LIST-KEYWORDS}}
  {List of macro lambda list keywords.
    }

\end{LIST}


%%%%%%%%%%%%%%%%%%%%%%%%%%%%%%%%%%%%%%%%%%%%%%%%%%
\subsection{Control Flow}
%%%%%%%%%%%%%%%%%%%%%%%%%%%%%%%%%%%%%%%%%%%%%%%%%%
\begin{LIST}{1cm}

  \IT{(\SO*{IF} \VAR{test} \VAR{then} \Op{\VAR{else}\DF{\NIL}})}
  {
    Return values of \retval{\VAR{then}} if \VAR{test} returns \T;
    return values of \retval{\VAR{else}} otherwise. 
  }

  \IT{(\MC*{COND} \OPn{(\VAR{test} \PROGN{\VAR{then}}\DF{\VAR{test}})})}
  {
  Return the \retval{values} of the first
  \OPn{\VAR{then}} whose \VAR{test} returns \T; return
  \retval{\NIL} if all \VAR{test}s return \NIL.
  }

  \IT{(\xorGOO{\MC*{WHEN}\\
      \MC*{UNLESS}}{\}} \VAR{test}
    \PROGN{\VAR{foo}}\DF{\NIL})}
  {
  Evaluate \VAR{foo}s and return \retval{their values} if \VAR{test}
  returns \T\ or \NIL, respectively. Return \retval{\NIL} otherwise. 
  }

  \IT{(\MC*{CASE} \VAR{test} \OPn{(\NEV{\VAR{keys}} \PROGN{\VAR{foo}})}
    \OP{(\xorGOO{\kwd*{OTHERWISE}\\
      \T}{\}} \PROGN{\VAR{bar}})\DF{\NIL}})}
  {
  Return the \retval{values} of the first \OPn{\VAR{foo}} one of whose \VAR{keys} is
  \kwd{eql} \VAR{test}.
  Return \retval{values of \VAR{bar}s} if no element of \VAR{keys} matches. 
  }

  \IT{(\Goo{\MC*{ECASE}\XOR\MC*{CCASE}} \VAR{test}
    \OPn{(\NEV{\VAR{keys}} \PROGN{\VAR{foo}})})}
  {
  Return the \retval{values} of the first \OPn{\VAR{foo}} one of whose \VAR{keys} is \kwd{eql}
  \VAR{test}. Signal non-correctable/correctable \kwd{type-error} and
  return \retval{\NIL} if no element of \VAR{keys} matches. 
  }

  \IT{(\MC*{AND} \OPn{\VAR{form}}\DF{\T})}
  {
  Evaluate \VAR{form}s from left to
  right. Immediately return \retval{\NIL} if one \VAR{form}'s value is \NIL. Return
  \retval{values of last \VAR{form}} otherwise.
  }

  \IT{(\MC*{OR} \OPn{\VAR{form}}\DF{\NIL})}
  {
  Evaluate \VAR{form}s from left to
  right. Immediately return \retval{primary value} of first
  non-\NIL-evaluating form, or \retval{all values} if last \VAR{form}
  is reached. Return \retval{\NIL} if no \VAR{form} returns \T.
  }

  \IT{(\SO*{PROGN} \OPn{\VAR{form}}\DF{\NIL})}
  {\label{:progn}
    Evaluate \VAR{form}s sequentially. Return
    \retval{values of last \VAR{form}}.
  }

  \IT{(\xorGOO{\MC*{PROG}\\
      \MC*{PROG\A}}{\}}
    (\orGOO{%
      \VAR{var}\\
      (\VAR{var } \Op{\VAR{value}\DF{\NIL}})}{\}^{\!\!*}})
    \OPn{(\kwd{declare} \OPn{\NEV{\VAR{decl}}})} 
    \xorGOO{\NEV{\VAR{tag}}\\\VAR{form}}{\}^{\!\!*}})}
  {
  Evaluate \SO{TAGBODY}-like body with \VAR{var}s locally bound (in
  parallel or sequentially, respectively) to \VAR{value}s. Return
  \retval{\NIL} or explicitly \retval{\MC{return}ed values}.
  Implicitly, the whole form is a \SO{block} named \NIL. 
  }

  \IT{\arrGOO{%
      (\SO*{MULTIPLE-VALUE-PROG1} \VAR{ form-r} \OPn{\VAR{ form}})\\
      (\MC*{PROG1} \VAR{ form-r} \OPn{\VAR{ form}})\\
      (\MC*{PROG2} \VAR{ form-a} \VAR{ form-r} \OPn{\VAR{ form}})}{.}}
  {
  Evaluate forms in order. Return \retval{values/1st value},
  respectively, of \VAR{form-r}. 
  }

  \IT{(\SO*{PROGV} \VAR{symbols} \VAR{values} \PROGN{\VAR{form}})}
  {
  Evaluate \VAR{form}s with \VAR{symbols} dynamically bound to \VAR{values} or
  \NIL. Return \retval{values of \VAR{form}s}.
  }

  \IT{(\SO*{UNWIND-PROTECT} \VAR{protected} \OPn{\VAR{cleanup}})}
  {
  Evaluate \VAR{protected} and then, no matter how control leaves
  \VAR{protected}, \VAR{cleanup}s. Return \retval{values of
    \VAR{protected}}. 
  }

  \IT{(\MC*{DESTRUCTURING-BIND} \VAR{destruct-$\lambda$} \VAR{bar}  \OPn{(\kwd{declare}
      \OPn{\NEV{\VAR{decl}}})} \PROGN{\VAR{form}})}
  {
  Evaluate \VAR{form}s with variables from tree \VAR{destruct-$\lambda$} bound to
  corresponding elements of tree \VAR{bar}, and return \retval{their
    values}. \VAR{destruct-$\lambda$} resembles \VAR{macro-$\lambda$}
  (section \ref{section:Macros}), but without any \kwd{\&environment}
  clause. 
  }

  \IT{(\MC*{MULTIPLE-VALUE-BIND} (\OPn{\NEV{\VAR{var}}}) \VAR{values-form}
    \OPn{(\kwd{declare} \OPn{\NEV{\VAR{decl}}})} \PROGN{\VAR{body-form}})}
  {
  Evaluate
  \VAR{body-form}s with \VAR{var}s lexically bound to the return values of
  \VAR{values-form}. Return \retval{values of \VAR{body-form}s}.
  }

  \IT{(\xorGOO{\SO*{LET}\\ 
      \SO*{LET\A}}{\}}(%
    \orGOO{\VAR{name}\\
      (\VAR{name }\Op{\VAR{value}\DF{\NIL}})}{\}^{\!\!*}}) 
    \OPn{(\kwd{declare} \OPn{\NEV{\VAR{decl}}})} 
    \PROGN{\VAR{form}})}
  {
  Evaluate \VAR{form}s with \VAR{name}s lexically bound (in parallel or
  sequentially, respectively) to \VAR{value}s. Return
  \retval{values of \VAR{form}s}. 
  }

  \IT{(\SO*{LOCALLY} \OPn{(\kwd{declare} \OPn{\NEV{\VAR{decl}}})}
    \PROGN{\VAR{form}})}
  {
  Evaluate \VAR{form}s in a lexical invironment with declarations
  \VAR{decl} in effect. Return \retval{values of \VAR{form}s}. 
  }

  \IT{(\SO*{BLOCK} \VAR{name} \PROGN{\VAR{form}})}
  { 
    Evaluate \VAR{form}s with lexical scope and dynamic extent, and
    return \retval{their values} unless interrupted by
    \SO{RETURN-FROM}. 
  }

  \IT{\arrGOO{(\SO*{RETURN-FROM } \VAR{foo } \Op{\VAR{result}\DF{\NIL}})\\
      (\MC*{RETURN } \Op{\VAR{result}\DF{\NIL}})}{.}}
  {
  Have nearest enclosing \SO{block} named \VAR{foo}/named \NIL,
  repectively, return with values of \VAR{result}.  
  }

  \IT{(\SO*{TAGBODY} \Goos{\NEV{\VAR{tag}}\XOR\VAR{form}})}
  {
  Evaluate \VAR{form}s. \VAR{tag}s (symbols or integers) have lexical
  scope and dynamic extent, and are targets for \SO{GO}. Return
  \retval{\NIL}. 
  }

  \IT{(\SO*{GO} \VAR{tag})}
  {
    Within the innermost enclosing \SO{tagbody}, jump to a tag \kwd{eql} \VAR{tag}.
  }

  \IT{(\SO*{CATCH} \VAR{tag} \PROGN{\VAR{form}})}
  {
    Evaluate \VAR{form}s and return \retval{their values} unless
    interrupted by \SO{THROW}. 
  }

  \IT{(\SO*{THROW} \VAR{tag} \VAR{form})}
  {Have the nearest dynamically
  enclosing \SO{CATCH} with a tag \FU{eq} \VAR{tag} return with the
  values of \VAR{form}.
  }

  \IT{(\FU*{SLEEP} \VAR{n})}
  {Wait \VAR{n} seconds, return \retval{\NIL}.
    }

\end{LIST}


%%%%%%%%%%%%%%%%%%%%%%%%%%%%%%%%%%%%%%%%%%%%%%%%%%
\subsection{Iteration}
%%%%%%%%%%%%%%%%%%%%%%%%%%%%%%%%%%%%%%%%%%%%%%%%%%

\begin{LIST}{1cm}
  
  \IT{(\xorGOO{\MC*{DO}\\
      \MC*{DO\A}}{\}}
    (\xorGOO{%
      \VAR{var}\\
      (\VAR{var } \OP{ \VAR{start } \Op{\VAR{step}}})}{\}^{\!\!*}})
    (\VAR{stop} \PROGN{\VAR{result}})
    \OPn{(\kwd{declare} \OPn{\NEV{\VAR{decl}}})} 
    \xorGOO{\NEV{\VAR{tag}}\\\VAR{form}}{\}^{\!\!*}})}
  {
  Evaluate \SO{TAGBODY}-like body with \VAR{var}s successively bound according
  to the values of the corresponding \VAR{start} and \VAR{step}
  forms. \VAR{var}s are bound in parallel/sequentially, respectively.
  Stop iteration when \VAR{stop} is \T. Return \retval{values of
    \OPn{\VAR{result}}}. Implicitly, the whole form is a \SO{block}
  named \NIL.
  }

  \IT{(\MC*{DOTIMES} (\VAR{var} \VAR{i} \Op{\VAR{result}})
    \OPn{(\kwd{declare} \OPn{\NEV{\VAR{decl}}})}
    \Goos{\NEV{\VAR{tag}}\XOR\VAR{form}})}
  {
  Evaluate \SO{TAGBODY}-like body with \VAR{var} successively bound
  to integers from 0 to $i - 1$. Upon evaluation of
  \retval{\VAR{result}}, \VAR{var} is \VAR{i}. Implicitly, the whole
  form is a \SO{block} named \NIL. 
  }

  \IT{(\MC*{DOLIST }(\VAR{var} \VAR{list} \Op{\VAR{result}})
    \OPn{(\kwd{declare} \OPn{\NEV{\VAR{decl}}})}
    \Goos{\NEV{\VAR{tag}}\XOR\VAR{form}})}
  {
  Evaluate \SO{TAGBODY}-like body with \VAR{var} successively bound
  to the elements of \VAR{list}. Upon evaluation of
  \retval{\VAR{result}}, \VAR{var} is \NIL. Implicitly, the whole form
  is a \SO{block} named \NIL. 
  }


\end{LIST}

%%%%%%%%%%%%%%%%%%%%%%%%%%%%%%%%%%%%%%%%%%%%%%%%%%
\subsection{Loop Facility}
%%%%%%%%%%%%%%%%%%%%%%%%%%%%%%%%%%%%%%%%%%%%%%%%%%


\begin{LIST}{1cm}

  \IT{(\MC*{LOOP} \OPn{\VAR{form}})} 
  {
  Simple Loop. If \VAR{form}s do not contain any atomic Loop Facility
  keywords, evaluate them forever in an implicit \SO{block} named \NIL.
  }

  \IT{(\MC*{LOOP} \OPn{\VAR{form}})}
  {
  Loop Facility. For Loop Facility keywords
  see below and Figure \ref{loop-overview}.
  }

  \begin{LIST}{.5cm}
    
    \IT{\LKWD*{named} \VAR{n}\DF{\NIL}}
    {
    Give \MC{loop}'s implicit \SO{block} a name.
    }

    \IT{\Goop{\LKWD*{with}
        \xorGOO{%
          \VAR{var-s}\\
          (\OPn{\VAR{var-s}})}{\}}
        \Op{\VAR{d-type}} \LKWD*{=} \VAR{foo}} \Goos{\LKWD*{and}
        \xorGOO{%
          \VAR{var-p}\\
          (\OPn{\VAR{var-p}})}{\}}
        \Op{\VAR{d-type}} \LKWD*{=} \VAR{bar}}
      \penalty-10
      where destructuring type specifier \VAR{d-type} has the form
      \penalty-5
      \GOO{\kwd{fixnum}\XOR
      \kwd{float}\XOR
      \T\XOR
      \NIL\XOR
      \GOo{%
      \LKWD*{of-type}
      \xorGOO{\VAR{type}\\
      (\OPn{\VAR{type}})}{\}}}}
    }
    {
    Initialize (possibly trees of) local variables \VAR{var-s}
    sequentially and \VAR{var-p} in parallel. 
    }

    \IT{\Goo{\LKWD*{initially}\XOR\LKWD*{finally}} \RP{\VAR{form}}}
    {
    Evaluate \VAR{form}s before begin, or after end, respectively, of iterations.
    }

    \IT{\GOop{\Goo{\LKWD*{for}\XOR\LKWD*{as}}
        \xorGOO{%
          \VAR{var-s}\\
          (\OPn{\VAR{var-s}})}{\}}
        \Op{\VAR{d-type}}} 
      \GOos{\LKWD*{and}
        \xorGOO{%
          \VAR{var-p}\\
          (\OPn{\VAR{var-p}})}{\}}
        \Op{\VAR{d-type}}}}
    {
    Begin of iteration control clauses. Initialize and step (possibly
    trees of) local variables \VAR{var-s} sequentially and \VAR{var-p}
    in parallel. Destructuring type specifier \VAR{d-type} as with \LKWD{with}.
    }

    \begin{LIST}{.5cm}

      \IT{\Goo{\LKWD*{upfrom}\XOR\LKWD*{from}\XOR\LKWD*{downfrom}}
        \VAR{start}}
      {
      Start stepping with \VAR{start}
      }

      \IT{\Goo{\LKWD*{upto}\XOR\LKWD*{downto}\XOR\LKWD*{to}\XOR\LKWD*{below}\XOR\LKWD*{above}}
        \VAR{form}}
      {
      Specify \VAR{form} as the end value for stepping.
      }

      \IT{\Goo{\LKWD*{in}\XOR\LKWD*{on}} \VAR{list}}
      {
      Bind \VAR{var} to successive elements/tails, respectively, of \VAR{list}.
      }

      \IT{\LKWD*{by} \Goo{\VAR{step}\DF{\LIT{1}}\XOR\VAR{function}\DF{\FU{cdr}}}}
      {
      Specify the (positive) decrement or increment or the
      \VAR{function} of one argument returning the next part of the list.
      }

      \IT{\LKWD*{=} \VAR{foo} \Op{\LKWD*{then}
          \VAR{bar}\DF{\VAR{foo}}}}
      {
      Bind \VAR{var} in the first iteration to  \VAR{foo} and later to \VAR{bar}.
      }

      \IT{\LKWD*{across} \VAR{vector}}
      {
      Bind \VAR{var} to successive elements of \VAR{vector}.
      }

      \IT{\LKWD*{being} \Goo{\LKWD*{the}\XOR\LKWD*{each}}}
      {
      Iterate over a hash table or a package.
      }

      \begin{LIST}{.5cm}

        \IT{\Goo{\LKWD*{hash-key}\XOR\LKWD*{hash-keys}} \Goo{\LKWD*{of}\XOR\LKWD*{in}} \VAR{hash-table}
          \Op{\LKWD*{using} (\LKWD*{hash-value} \VAR{value})}}
        {
        Bind \VAR{var} successively to the keys of \VAR{hash-table}; bind \VAR{value} to corresponding values.
        }

        \IT{\Goo{\LKWD*{hash-value}\XOR\LKWD*{hash-values}} \Goo{\LKWD*{of}\XOR\LKWD*{in}} \VAR{hash-table}
          \Op{\LKWD*{using} (\LKWD*{hash-key} \VAR{key})}}
        {
        Bind \VAR{var} successively to the values of \VAR{hash-table}; bind \VAR{key} to corresponding keys.
        }

        \IT{\Goo{\LKWD*{symbol}\XOR\LKWD*{symbols}\XOR\LKWD*{present-symbol}\XOR\LKWD*{present-symbols}\XOR\LKWD*{external-symbol}\XOR\LKWD*{external-symbols}}
          \Op{\Goo{\LKWD*{of}\XOR\LKWD*{in}}
            \VAR{package}\DF{\V{\A package\A}}}}
        {
        Bind \VAR{var} successively to the accessible symbols, or the present
        symbols, or the external symbols respectively, of
        \VAR{package}. 
        }

      \end{LIST}
    \end{LIST}
    
\begin{figure}%
  \begin{center}%
    \begin{sideways}%
      (%  
      \arraycolsep0pt
      \(
      \text{\kwd{loop}}\text{ }
      \Op{
        \text{\LKWD{named} \VAR{n}\DF{\NIL}}
      }
      \left\{
      \begin{array}{l}
        \text{\LKWD{with}
          \xorGOO{%
              \VAR{var}\\
              (\OPn{\VAR{var}})}{\}}
          \Op{\VAR{d-type}} \LKWD{=} \VAR{foo }}
        \{\text{\LKWD{and}
          \xorGOO{%
            \VAR{var}\\
            (\OPn{\VAR{var}})}{\}}
          \Op{\VAR{d-type}} \LKWD{=}
          \VAR{bar}}\}^{*} \\ \\[-2.4ex]
        \left.\!
        \begin{array}{l}
          \text{\LKWD{for}}  \\
          \text{\LKWD{as}} 
        \end{array}\right\}
        \boxed{
          \text{\xorGOO{%
              \VAR{var}\\
              (\OPn{\VAR{var}})}{\}}}%
          \Op{\VAR{d-type}}
          \left\{
          \begin{array}{l}
            \left.\!
            \begin{array}{l}
              \OP{
                \left\{
                \begin{array}{l}
                  \text{\LKWD{upfrom}}\\
                  \text{\LKWD{from}}
                \end{array}
                \right\}
                \text{\VAR{start}\DF{\LIT{0}}}
              }\text{ }
              \OP{
                \left\{
                \begin{array}{l}
                  \text{\LKWD{upto}}\\
                  \text{\LKWD{to}}\\
                  \text{\LKWD{below}}
                \end{array}
                \right\}
                \text{\VAR{form}}
              }\\
              \text{\LKWD{from} \VAR{start}}
              \left\{
              \begin{array}{l}
                \text{\LKWD{downto}}\\
                \text{\LKWD{above}}
              \end{array}
              \right\}              
              \text{\VAR{form }}\\
              \text{\LKWD{downfrom} \VAR{start}}\text{ }
              \OP{
                \left\{
                \begin{array}{l}
                  \text{\LKWD{downto}}\\
                  \text{\LKWD{to}}\\
                  \text{\LKWD{above}}
                \end{array}
                \right\}            
                \text{\VAR{form}}
              }
            \end{array}
            \right\}
            \Op{   
              \text{\LKWD{by} \VAR{step}\DF{\LIT{1}}}
            }\\
            \left.\!
            \begin{array}{l}
              \text{\LKWD{in}} \\
              \text{\LKWD{on}}
            \end{array}
            \right\}
            \text{\VAR{list}}\text{ }
            \Op{
              \text{\LKWD{by} \VAR{function}\DF{\kwd{cdr}}}
            }\\
            \text{\LKWD{=} \VAR{foo }} 
            \Op{
              \text{\LKWD{then} \VAR{bar}\DF{\VAR{foo}}}
            }\\
            \text{\LKWD{across} \VAR{vector}}\\
            \text{\LKWD{being}}
            \left\{
            \begin{array}{l}
              \text{\LKWD{the}}\\          
              \text{\LKWD{each}}
            \end{array}
            \right\}%
            \left\{
            \begin{array}{{l}}
              \left.\!
              \begin{array}{l}
                \left.\!
                \begin{array}{l}
                  \text{\LKWD{hash-key}}\\
                  \text{\LKWD{hash-keys}}
                \end{array}
                \right\}%
                \left\{
                \begin{array}{l}
                  \text{\LKWD{of}}\\
                  \text{\LKWD{in}}
                \end{array}
                \right\}
                \text{\VAR{hash }} 
                \Op{
                  \text{\LKWD{using} (\LKWD{hash-value} \VAR{v})}%
                }\\
                \left.\!
                \begin{array}{l}
                  \text{\LKWD{hash-value}}\\
                  \text{\LKWD{hash-values}}
                \end{array}
                \right\}%
                \left\{
                \begin{array}{l}
                  \text{\LKWD{of}}\\
                  \text{\LKWD{in}}
                \end{array}
                \right\}
                \text{\VAR{hash }}
                \Op{
                  \text{\LKWD{using} (\LKWD{hash-key} \VAR{k})}%
                }\\
              \end{array}
              \right.\\
              \left.\!
              \begin{array}{l}
                \text{\LKWD{symbol\Op{s}}}\\
                \text{\LKWD{present-symbol\Op{s}}}\\
                \text{\LKWD{external-symbol\Op{s}}}
              \end{array}
              \right\}%
              \OP{
                \left\{
                \begin{array}{l}
                  \text{\LKWD{of}}\\
                  \text{\LKWD{in}}
                \end{array}
                \right\} \text{\VAR{package}\DF{\V{\A package\A}}}%
              }%
            \end{array}\!\!%
            \right.%
          \end{array}%
          \right\}_{\displaystyle{\mathbb{F}_0}}
                }%boxed
                \left\{
                \text{\LKWD{and} }\boxed{\mathbb{F}_i\!}
                \right\}^{\!*}\\
                \\[-2.4ex]
                \left.\!%
                \begin{array}{l}
                  \text{\LKWD{initially}} \\
                  \text{\LKWD{finally}} 
                \end{array} 
                \right\} 
                \text{\VAR{form}}^{+} \\
                
      \end{array}
      \right\}^{\!\!\displaystyle{*}}
      %Bottleneck
      \left\{
      \begin{array}{l}
        \text{\LKWD{repeat} \VAR{num}}\\
        \left.\!
        \begin{array}{l}
          \text{\LKWD{while}}\\
          \text{\LKWD{until}}\\
          \text{\LKWD{always}}\\
          \text{\LKWD{never}}\\
          \text{\LKWD{thereis}}
        \end{array}
        \right\} 
        \text{\VAR{test}}\\
        \boxed{
          \left.\!
          \begin{array}{l}
            \left.\!
            \begin{array}{l}
              \text{\LKWD{do}}\\
              \text{\LKWD{doing}}
            \end{array}
            \right\}
            \text{\VAR{form}}^{+}\\
            \text{\LKWD{return}}
            \left\{
            \begin{array}{l}
              \text{\VAR{form}}\\
              \text{\LKWD{it}}
            \end{array}
            \right.\\
            \left.\!
            \begin{array}{l}
              \text{\LKWD{collect\Op{ing}}}\\
              \text{\LKWD{append\Op{ing}}}\\
              \text{\LKWD{nconc\Op{ing}}}
            \end{array}
            \right\}
            \left\{
            \begin{array}{l}
              \text{\VAR{form}}\\
              \text{\LKWD{it}}
            \end{array}
            \right\} \Op{\text{\LKWD{into }\VAR{list}}}\\
            \left.\!
            \begin{array}{l}
              \text{\LKWD{count\Op{ing}}}\\
              \text{\LKWD{sum\Op{ming}}}\\
              \text{\LKWD{maximize}}\\
              \text{\LKWD{maximizing}}\\
              \text{\LKWD{minimize}}\\
              \text{\LKWD{minimizing}}
            \end{array}
            \right\}
            \left\{
            \begin{array}{l}
              \text{\VAR{form}}\\
              \text{\LKWD{it}}
            \end{array}
            \right\} 
                \Op{\text{\LKWD{into }\VAR{num}}}\text{ }
                \Op{\text{\VAR{type}}}\\
                \left.\!
                \begin{array}{l}
                  \text{\LKWD{if}}\\
                  \text{\LKWD{when}}\\
                  \text{\LKWD{unless}}
                \end{array}
                \right\}%
                \text{\VAR{test }}
                \boxed{\mathbb{C}_i\!}\{\text{\LKWD{and}}\,
                \boxed{\mathbb{C}_j\!}\}^{*}\Op{\text{\LKWD{else}}\,
                  \boxed{\mathbb{C}_k\!}\{\text{\LKWD{and}}\,
                  \boxed{\mathbb{C}_l\!}\}^{*}}
                \Op{\text{\LKWD{end}}}\!\!\!\!\!\!
          \end{array}
          \right._{\!\!\!\!\displaystyle{\mathbb{C}_0}}
        }%boxed
        \\
        \\[-2.4ex]
        \left.\!
        \begin{array}{l}
          \text{\LKWD{initially}} \\
          \text{\LKWD{finally}} 
        \end{array} 
        \right\}
        \text{\VAR{form}}^{+}
      \end{array}
      \right\}^{\displaystyle{\!\!*}}
      \)%
      \!)%
    \end{sideways}%
  \end{center}%
  \vspace{-3em}
  \raisebox{0em}[0em][0em]{\parbox[b]{4cm}{\caption{\protect\raggedright Loop Facility, Overview.\label{loop-overview}}}}
\end{figure}

    \IT{\Goo{\LKWD*{do}\XOR\LKWD*{doing}} \RP{\VAR{form}}}
    {
    Evaluate \VAR{form}s in every iteration.
    }

    \IT{\LKWD*{it}}
    {
      Value of \VAR{test} form of an enclosing \LKWD{if},
      \LKWD{when}, or \LKWD{unless} clause.
    }

    \IT{\LKWD*{return} \Goo{\VAR{form}\XOR\LKWD*{it}}}
    {
    Return immediately, skipping any \LKWD{finally} parts, with values of \VAR{form} or \LKWD{it}.
    }

    \IT{\Goo{\LKWD*{collect}\XOR\LKWD*{collecting}}
      \Goo{\VAR{form}\XOR\LKWD*{it}} \Op{\LKWD*{into} \VAR{list}}}
    {
    Collect values of \VAR{form} or \LKWD{it} into \VAR{list}. If no
    \VAR{list} is given, collect into an anonymous list which is
    returned after termination. 
    }

    \IT{\Goo{\LKWD*{append}\XOR\LKWD*{appending}\XOR\LKWD*{nconc}\XOR\LKWD*{nconcing}}
      \Goo{\VAR{form}\XOR\LKWD*{it}} \Op{\LKWD*{into}
        \VAR{list}}}
    {
    Concatenate values of \VAR{form} or \LKWD{it}, which should be
    lists, into \VAR{list} by the means of \FU{append} or \FU{nconc},
    respectively.  If no \VAR{list} is given, collect into an
    anonymous list which is returned after termination. 
    }

    \IT{\Goo{\LKWD*{count}\XOR\LKWD*{counting}}
      \Goo{\VAR{form}\XOR\LKWD*{it}} \Op{\LKWD*{into}
        \VAR{n}} \Op{\VAR{type}}}
    {
    Count the number of times the value of \VAR{form} or of \LKWD{it} is \T.
    If no \VAR{n} is given, count into an anonymous variable
    which is returned after termination.
    }

    \IT{\Goo{\LKWD*{sum}\XOR\LKWD*{summing}}
      \Goo{\VAR{form}\XOR\LKWD*{it}} \Op{\LKWD*{into}
        \VAR{sum}} \Op{\VAR{type}}} 
    {
    Calculate the sum of the primary values of \VAR{form} or of \LKWD{it}.
    If no \VAR{sum} is given, sum into an anonymous variable
    which is returned after termination.
    }

    \IT{\Goo{\LKWD*{maximize}\XOR\LKWD*{maximizing}\XOR
        \LKWD*{minimize}\XOR
        \LKWD*{minimizing}} \Goo{\VAR{form}\XOR\LKWD*{it}} \Op{\LKWD*{into}
        \VAR{max-min}} \Op{\VAR{type}}} 
    {
    Determine the maximum or minimum, respectively, of the primary values of
    \VAR{form} or of \LKWD{it}. 
    If no \VAR{max-min} is given, use an anonymous variable
    which is returned after termination.
    }

    \IT{\Goo{\LKWD*{if}\XOR\LKWD*{when}\XOR\LKWD*{unless}} \VAR{ test}
      \VAR{i-form} \Goos{\LKWD*{and} 
        \VAR{j-form}} \Op{\LKWD*{else} \VAR{k-form} \Goos{\LKWD*{and} 
          \VAR{l-form}}} \Op{\LKWD*{end}}} 
    {
    If \VAR{test} returns \T, \T, or \NIL, respectively, evaluate
    \VAR{i-form} and \VAR{j-form}s; otherwise, evaluate \VAR{k-form}
    and \VAR{l-form}s. Inside \VAR{i-form} and \VAR{k-form}, the value
    of \VAR{test} is accessible by \LKWD*{it}. 
    }

    \IT{\LKWD*{repeat} \VAR{num}}
    {
    Terminate \MC{loop} after \VAR{num} iterations; \VAR{num} is evaluated once.
    }

    \IT{\Goo{\LKWD*{while}\XOR\LKWD*{until}} \VAR{test}}
    {
    Continue iteration until \VAR{test} returns \NIL\ or \T, respectively.
    }

    \IT{\Goo{\LKWD*{always}\XOR\LKWD*{never}} \VAR{test}}
    {
    Terminate \MC{loop} returning \NIL\ and skipping any
    \LKWD{finally} parts as soon as \VAR{test} is \NIL\ or \T,
    respectively. Otherwise continue \MC{loop} with its default return
    value set to \T. 
    }

    \IT{\LKWD*{thereis} \VAR{test}}
    {
    Terminate \MC{loop} when \VAR{test} is \T\ and return value of
    \VAR{test}, skipping any \LKWD{finally} parts. Otherwise continue
    \MC{loop} with its default return value set to \NIL. 
    }

    \IT{\LKWD*{loop-finish}}
    {
    Terminate \MC{loop} immediately executing any \LKWD{finally}
    clauses and returning any accumulated results.
    }
  \end{LIST}
\end{LIST}




% LocalWords:  pt

%%% Local Variables: 
%%% mode: latex
%%% TeX-master: "clqr"
%%% End: 

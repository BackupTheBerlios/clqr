% Copyright (C) 2008, 2009, 2010, 2011 Bert Burgemeister
%
% Permission is granted to copy, distribute and/or modify this
% document under the terms of the GNU Free Documentation License,
% Version 1.2 or any later version published by the Free Software
% Foundation; with no Invariant Sections, no Front-Cover Texts and
% no Back-Cover Texts. For details see file COPYING.
%

%%%%%%%%%%%%%%%%%%%%%%%%%%%%%%%%%%%%%%%%%%%%%%%%%%
\section{Conses} 
%%%%%%%%%%%%%%%%%%%%%%%%%%%%%%%%%%%%%%%%%%%%%%%%%%

%%%%%%%%%%%%%%%%%%%%%%%%%%%%%%%%%%%%%%%%%%%%%%%%%%
\subsection{Predicates} 
%%%%%%%%%%%%%%%%%%%%%%%%%%%%%%%%%%%%%%%%%%%%%%%%%%
\begin{LIST}{1cm}

  \IT{\arrGOO{(\FU*{CONSP} \VAR{ foo})\\
      (\FU*{LISTP} \VAR{ foo})}{.}}
  {
    Return \retval{\T} if \VAR{foo} is of indicated type.
    }

  \IT{\arrGOO{(\FU*{ENDP} \VAR{ list})\\
      (\FU*{NULL} \VAR{ foo})}{.}\qquad\qquad}
  {
  Return \retval{\T} if \VAR{list}/\VAR{foo} is \NIL.
  }

  \IT{(\FU*{ATOM} \VAR{foo})\qquad\qquad}
  {Return \retval{\T} if \VAR{foo} is not a
  \kwd{cons}. 
  }

  \IT{(\FU*{TAILP} \VAR{foo} \VAR{list})}
  {
  Return \retval{\T} if \VAR{foo} is a tail of \VAR{list}.
  }

  \IT{(\FU*{MEMBER} \VAR{foo} \VAR{list}
    \orGOO{\xorGOO{%
        \kwd{:test} \VAR{ function}\DF{\kwd{\#'eql}}\\
        \kwd{:test-not} \VAR{ function}}{.}\\
      \kwd{:key} \VAR{ function}}{\}})}
  {
  Return \retval{tail of \VAR{list}} starting with
  its first element matching \VAR{foo}. Return \retval{\NIL} if
  there is no such element.
  }

  \IT{(\xorGOO{\FU*{MEMBER-IF}\\
      \FU*{MEMBER-IF-NOT}}{\}}
    \VAR{test} \VAR{list}
    \Op{\kwd{:key} \VAR{function}})}
  {
  Return \retval{tail of \VAR{list}} starting with
  its first element satisfying \VAR{test}. Return \retval{\NIL} if
  there is no such element.
  }

  \IT{(\FU*{SUBSETP} \VAR{list-a} \VAR{list-b} 
    \orGOO{\xorGOO{%
        \kwd{:test} \VAR{ function}\DF{\kwd{\#'eql}}\\
        \kwd{:test-not} \VAR{ function}}{.}\\
      \kwd{:key} \VAR{ function}}{\}})}
  {
  Return \retval{\T} if \VAR{list-a} is a subset of
  \VAR{list-b}.
  }

\end{LIST}


%%%%%%%%%%%%%%%%%%%%%%%%%%%%%%%%%%%%%%%%%%%%%%%%%%
\subsection{Lists} 
%%%%%%%%%%%%%%%%%%%%%%%%%%%%%%%%%%%%%%%%%%%%%%%%%%
\begin{LIST}{1cm}

  \IT{(\FU*{CONS} \VAR{foo} \VAR{bar})}
  {
  Return new cons \retval{(\VAR{foo} \kwd {.} \VAR{bar})}.
  }

  \IT{(\FU*{LIST} \OPn{\VAR{foo}})}
  {Return \retval{list of \VAR{foo}s}.
    }

  \IT{(\FU*{LIST\A} \RP{\VAR{foo}})}
  {Return \retval{list of \VAR{foo}s}
  with last \VAR{foo} becoming cdr of last cons. Return
  \retval{\VAR{foo}} if only one \VAR{foo} given.
  }

  \IT{(\FU*{MAKE-LIST} \VAR{num} \Op{\kwd{:initial-element}
      \VAR{foo}\DF{\NIL}})}
  {
    New \retval{list} with \VAR{num} elements set to \VAR{foo}.
  }

  \IT{(\FU*{LIST-LENGTH} \VAR{list})}
  {\retval{Length} of \VAR{list};
  \retval{\NIL} for circular \VAR{list}.
  }

  \IT{(\FU*{CAR} \VAR{list})\qquad\qquad}
  {
  \retval{Car of \VAR{list}} or \retval{\NIL} if \VAR{list} is
  \NIL. \kwd{SETF}able. 
  }

  \IT{\arrGOO{(\FU*{CDR} \VAR{ list})\qquad\\
      (\FU*{REST} \VAR{ list})}{.}}
  {
  \retval{Cdr of \VAR{list}} or \retval{\NIL} if \VAR{list}
  is \NIL. \kwd{SETF}able. 
  }

  \IT{(\FU*{NTHCDR} \VAR{n list})}
  {Return \retval{tail of \VAR{list}} after calling \FU{cdr} \VAR{n} times.
    }

  \IT{(\Goo{\FU*{FIRST}\XOR\FU*{SECOND}\XOR\FU*{THIRD}\XOR\FU*{FOURTH}\XOR\FU*{FIFTH}\XOR\FU*{SIXTH}\XOR\dots\XOR\FU*{NINTH}\XOR\FU*{TENTH}}
    \VAR{list})}
  {
  \index{SEVENTH}%
  \index{EIGHTH}%
  Return \retval{nth element of \VAR{list}} if any,
  or \retval{\NIL} otherwise. \kwd{SETF}able.
  }

  \IT{(\FU*{NTH} \VAR{n list})}
  {
    Zero-indexed \retval{\VAR{n}th element} of \VAR{list}. \kwd{setf}able.
  }

  \IT{(\FU{C}\VAR{X}\kwd{R} \VAR{list})}
  {
  \index{CAAR}%
  \index{CADR}%
  \index{CDAR}%
  \index{CDDR}%
  With \VAR{X} being one to four
  \kwd{a}s and \kwd{d}s representing \FU{CAR}s and \FU{CDR}s, e.g.
  (\FU{CADR} \VAR{bar}) is equivalent to (\FU{CAR} (\FU{CDR}
  \VAR{bar})).
  \kwd{SETF}able.
  }

  \IT{(\FU*{LAST} \VAR{list} \Op{\VAR{num}\DF{\LIT{1}}})}
  {
  Return list of \retval{last \VAR{num}
    conses} of \VAR{list}.
  }

  \IT{(\xorGOO{\FU*{BUTLAST }  \VAR{list}\\
      \FU*{NBUTLAST }  \DES{\VAR{list}}}{\}}
    \Op{\VAR{num}\DF{\LIT{1}}})}
  {
    \retval{\VAR{list}} excluding last \VAR{num}
    conses.
  }

  \IT{(\xorGOO{\FU*{RPLACA}\\
    \FU*{RPLACD}}{\}} \DES{\VAR{cons}} \VAR{object})}
  {
  Replace car, or cdr, respectively, of \retval{\VAR{cons}} with \VAR{object}.
  }

  \IT{(\FU*{LDIFF} \VAR{list} \VAR{foo})}
  {
  If \VAR{foo} is a tail of \VAR{list}, return \retval{preceding
    part of \VAR{list}}. Otherwise return \retval{\VAR{list}}.
  }

  \IT{(\FU*{ADJOIN} \VAR{foo} \VAR{list} 
    \orGOO{\xorGOO{%
        \kwd{:test} \VAR{ function}\DF{\kwd{\#'eql}}\\
        \kwd{:test-not} \VAR{ function}}{.}\\
      \kwd{:key} \VAR{ function}}{\}})}
  {Return \retval{\VAR{list}} if \VAR{foo} is
  already member of \VAR{list}. If not, return \retval{(\FU{CONS}
    \VAR{foo} \VAR{list})}.
  }

  \IT{(\MC*{POP} \DES{\VAR{place}})}
  {
    Set \VAR{place} to (\FU{CDR} \VAR{place}), return
    \retval{(\FU{CAR} \VAR{place})}.
  }

  \IT{(\MC*{PUSH} \VAR{foo} \DES{\VAR{place}})}
  {
    Set \VAR{place} to \retval{(\FU{cons} \VAR{foo} \VAR{place})}.
  }

  \IT{(\MC*{PUSHNEW} \VAR{foo} \DES{\VAR{place}}  
    \orGOO{\xorGOO{%
        \kwd{:test} \VAR{ function}\DF{\kwd{\#'eql}}\\
        \kwd{:test-not} \VAR{ function}}{.}\\
      \kwd{:key} \VAR{ function}}{\}})}
  {
    Set \VAR{place} to \retval{(\FU{adjoin} \VAR{foo} \VAR{place})}.
  }

  \IT{\arrGOO{(\FU*{APPEND } \Op{\OPn{\VAR{proper-list}} 
        \VAR{ foo}\DF{\NIL}})\\
      (\FU*{NCONC } \Op{\OPn{\DES{\VAR{non-circular-list}}} 
        \VAR{ foo}\DF{\NIL}})}{.}}
  {
    Return \retval{concatenated list} or, with only one argument,
    \retval{\VAR{foo}}. \VAR{foo} can be of any type.
  }

  \IT{\arrGOO{(\FU*{REVAPPEND} \VAR{ list} \VAR{ foo})\\
      (\FU*{NRECONC }
      \DES{\VAR{list}} \VAR{ foo})}{.}}
  {
    Return \retval{concatenated list} after reversing order in
    \VAR{list}.
  }

  \IT{(\xorGOO{\FU*{MAPCAR}\\
      \FU*{MAPLIST}}{\}} \VAR{function} \RP{\VAR{list}})}
  {
    Return \retval{list of return values} of \VAR{function}
    successively invoked with corresponding arguments, either cars or
    cdrs, respectively, from each \VAR{list}.
  }

  \IT{(\xorGOO{\FU*{MAPCAN}\\
      \FU*{MAPCON}}{\}} \VAR{function} \RP{\VAR{list}})}
  {
    Return list of \retval{concatenated return values} of
    \VAR{function} successively invoked with corresponding arguments,
    either cars or cdrs, respectively, from each
    \VAR{list}. \VAR{function} should return a list.
  }

  \IT{(\xorGOO{\FU*{MAPC}\\
      \FU*{MAPL}}{\}} \VAR{function} \RP{\VAR{list}})}
  {
    Return \retval{first \VAR{list}} after successively applying
    \VAR{function} to corresponding arguments, either cars or cdrs,
    respectively, from each \VAR{list}. \VAR{function} should have
    some side effects.
  }

  \IT{(\FU*{COPY-LIST} \VAR{list})}
  {
    Return \retval{copy} of \VAR{list} with shared elements.
  }

\end{LIST}


%%%%%%%%%%%%%%%%%%%%%%%%%%%%%%%%%%%%%%%%%%%%%%%%%%
\subsection{Association Lists} 
%%%%%%%%%%%%%%%%%%%%%%%%%%%%%%%%%%%%%%%%%%%%%%%%%%
\label{section:Association Lists}
\begin{LIST}{1cm}

  \IT{(\FU*{PAIRLIS} \VAR{keys} \VAR{values} \Op{\VAR{alist}\DF{\NIL}})}
  {
    Prepend to \retval{\VAR{alist}} an association list made from
    lists \VAR{keys} and \VAR{values}.
  }

  \IT{(\FU*{ACONS} \VAR{key} \VAR{value} \VAR{alist})}
  {
    Return \retval{\VAR{alist}} with a (\VAR{key} \kwd{.} \VAR{value})
    pair added.
  }

  \IT{\arrGOO{(\xorGOO{\FU*{ASSOC}\\
        \FU*{RASSOC}}{\}}
      \VAR{ foo} \VAR{ alist }
      \orGOO{\xorGOO{%
          \kwd{:test} \VAR{ test}\DF{\kwd{\#'eql}}\\
          \kwd{:test-not} \VAR{ test}}{.}\\
        \kwd{:key} \VAR{ function}
      }{\}})\\
    (\xorGOO{\FU*{ASSOC-IF}\Op{\kwd{-NOT}}\\
      \FU*{RASSOC-IF}\Op{\kwd{-NOT}}}{\}} \VAR{ test} \VAR{ alist }
      \Op{\kwd{:key} \VAR{ function}})}{.}}
  {\index{ASSOC-IF-NOT}\index{RASSOC-IF-NOT}%
    First \retval{cons} whose car, or cdr, respectively, satisfies
    \VAR{test}.
  }

  \IT{(\FU*{COPY-ALIST} \VAR{alist})}
  {
    Return \retval{copy} of \VAR{alist}.
  }

\end{LIST}


%%%%%%%%%%%%%%%%%%%%%%%%%%%%%%%%%%%%%%%%%%%%%%%%%%
\subsection{Trees} 
%%%%%%%%%%%%%%%%%%%%%%%%%%%%%%%%%%%%%%%%%%%%%%%%%%
\begin{LIST}{1cm}

  \IT{(\FU*{TREE-EQUAL} \VAR{foo} \VAR{bar} 
    \xorGOO{\kwd{:test} \VAR{ test}\DF{\kwd{\#'eql}}\\
      \kwd{:test-not} \VAR{ test}}{\}})}
  {
    Return \retval{\T} if trees \VAR{foo} and \VAR{bar} have same
    shape and leaves satisfying \VAR{test}.
  }

  \IT{(\xorGOO{\FU*{SUBST} \VAR{ new} \VAR{ old } \VAR{tree}\\
      \FU*{NSUBST} \VAR{ new} \VAR{ old } \DES{\VAR{tree}}}{\}}
    \orGOO{\xorGOO{%
        \kwd{:test} \VAR{ function}\DF{\kwd{\#'eql}}\\
        \kwd{:test-not} \VAR{ function}}{.}\\
      \kwd{:key} \VAR{ function}%
    }{\}})}
  {
    Make \retval{copy of \VAR{tree}} with each subtree or leaf
    matching \VAR{old} replaced by \VAR{new}.
  }

  \IT{(\xorGOO{\FU{SUBST-IF\Op{-NOT}} \VAR{ new} \VAR{ test } \VAR{tree}\\
      \FU{NSUBST-IF\Op{-NOT}} \VAR{ new} \VAR{ test } \DES{\VAR{tree}}}{\}} 
    \Op{\kwd{:key} \VAR{function}})}
  {
  \index{SUBST-IF}%
  \index{SUBST-IF-NOT}%
  \index{NSUBST-IF}%
  \index{NSUBST-IF-NOT}%
  Make \retval{copy of \VAR{tree}} with each subtree or leaf
  satisfying \VAR{test} replaced by \VAR{new}. 
  }

  \IT{(\xorGOO{\FU*{SUBLIS} \VAR{ association-list } \VAR{tree} \\
      \FU*{NSUBLIS} \VAR{ association-list } \DES{\VAR{tree}} }{\}}
    \orGOO{\xorGOO{%
        \kwd{:test} \VAR{ function}\DF{\kwd{\#'eql}}\\
        \kwd{:test-not} \VAR{ function}}{.}\\
      \kwd{:key} \VAR{ function}%
    }{\}})}
  {
  Make \retval{copy of \VAR{tree}} with each subtree or leaf matching
  a key in \VAR{association-list} replaced by that key's value.
  }

  \IT{(\FU*{COPY-TREE} \VAR{tree})} 
  {
    \retval{Copy of \VAR{tree}} with same shape and leaves.
  }

\end{LIST}


%%%%%%%%%%%%%%%%%%%%%%%%%%%%%%%%%%%%%%%%%%%%%%%%%%
\subsection{Sets} 
%%%%%%%%%%%%%%%%%%%%%%%%%%%%%%%%%%%%%%%%%%%%%%%%%%
\begin{LIST}{1cm}

  \IT{(\xorGOO{%
      \arrGOO{%
        \FU*{INTERSECTION}\\
        \FU*{SET-DIFFERENCE}\\
        \FU*{UNION}\\
        \FU*{SET-EXCLUSIVE-OR}%
      }{\}} \VAR{ a} \VAR{ b}\\
      \arrGOO{%
        \FU*{NINTERSECTION}\\
        \FU*{NSET-DIFFERENCE}%
      }{\}} \text{ }\DES{\VAR{a}}\text{ } \VAR{b}\\
      \arrGOO{%
        \FU*{NUNION}\\
        \FU*{NSET-EXCLUSIVE-OR}%
      }{\}} \text{ }\DES{\VAR{a}}\text{ } \DES{\VAR{b}}\text{ }
    }{\}}
    \orGOO{\xorGOO{%
        \kwd{:test} \VAR{ function}\DF{\kwd{\#'eql}}\\
        \kwd{:test-not} \VAR{ function}}{.}\\
      \kwd{:key} \VAR{ function}}{\}})}
  {
  Return \retval{$a\cap b$}, \retval{$a\setminus b$}, \retval{$a\cup b$}, or
  \retval{$a\,\triangle\, b$}, respectively, of lists \VAR{a} and \VAR{b}.
  }

\end{LIST}


%%% Local Variables: 
%%% mode: latex
%%% TeX-master: "clqr"
%%% End: 

%%%%%%%%%%%%%%%%%%%%%%%%%%%%%%%%%%%%%%%%%%%%%%%%%%
\section{Input/Output} 
%%%%%%%%%%%%%%%%%%%%%%%%%%%%%%%%%%%%%%%%%%%%%%%%%%

%%%%%%%%%%%%%%%%%%%%%%%%%%%%%%%%%%%%%%%%%%%%%%%%%%
\subsection{Predicates} 
%%%%%%%%%%%%%%%%%%%%%%%%%%%%%%%%%%%%%%%%%%%%%%%%%%

\begin{LIST}{1cm}

  \IT{\arrGOO{(\FU*{STREAMP} \VAR{ foo})\\
      (\FU*{PATHNAMEP} \VAR{ foo})\\
      (\FU*{READTABLEP} \VAR{ foo})}{\}}}
  Return \retval{\T} if \VAR{foo} is of type \kwd{stream},
  \kwd{pathname}, or \kwd{readtable}, resp.

  \IT{\arrGOO{(\FU*{INPUT-STREAM-P} \VAR{ stream})\\
      (\FU*{OUTPUT-STREAM-P} \VAR{ stream})\\
      (\FU*{INTERACTIVE-STREAM-P} \VAR{ stream})\\
      (\FU*{OPEN-STREAM-P} \VAR{ stream})}{\}}}
  Return \retval{\T} if \VAR{stream} is for input, for output,
  interactive, or open, resp.

  \IT{(\FU*{PATHNAME-MATCH-P} \VAR{path} \VAR{wildcard})}
  \retval{\T} if \VAR{path} matches \VAR{wildcard}.

  \IT{(\FU*{WILD-PATHNAME-P} \VAR{path}
    \OP{\Goo{\kwd{:host}\XOR\kwd{:device}\XOR
        \kwd{:directory}\XOR\kwd{:name}\XOR\kwd{:type}
        \XOR\kwd{:version}\XOR\NIL}})}
  Return \retval{\T} if indicated component in \VAR{path} is wildcard.

\end{LIST}


%%%%%%%%%%%%%%%%%%%%%%%%%%%%%%%%%%%%%%%%%%%%%%%%%%
\subsection{Reader} 
%%%%%%%%%%%%%%%%%%%%%%%%%%%%%%%%%%%%%%%%%%%%%%%%%%

\begin{LIST}{1cm}

  \IT{(\xorGOO{\FU*{Y-OR-N-P}\\
      \FU*{YES-OR-NO-P}}{\}} \OP{\VAR{control} \Op{\VAR{args}}})}
  Ask user a question and return \retval{\T} or \retval{\NIL}
  depending on their answer. See p.\ \pageref{section:Format}, \kwd{format}, for \VAR{control} and \VAR{args}.

  \IT{(\xorGOO{\FU*{READ}\\
      \FU*{READ-PRESERVING-WHITESPACE}}{\}}
    \OP{\VAR{stream}\DF{\T} \OP{\VAR{eof-err}\DF{\T}
        \Op{\VAR{eof-val}\DF{\NIL} \Op{\VAR{recursive}\DF{\NIL}}}}})} 
  Read printed representation of \retval{object}.

  \IT{(\FU*{READ-DELIMITED-LIST} \VAR{char}
    \OP{\VAR{stream}\DF{\kwd{*standard-input*}}
      \Op{\VAR{recursive}\DF{\NIL}}})}
  Continue reading until encountering \VAR{char}. Return \retval{list}
  of objects read.

  \IT{(\FU*{PRINT-UNREADABLE-OBJECT} (\VAR{object} \VAR{stream}
    \Op{\kwd{:type} \VAR{bool}} \Op{\kwd{:identity} \VAR{bool}})
    \OPn{form})} \retval{\NIL}

  \IT{(\FU*{READ-CHAR} \OP{\VAR{stream}\DF{\T}
      \OP{\VAR{eof-err}\DF{\T} \Op{\VAR{eof-val}\DF{\NIL}
          \Op{\VAR{recursive}\DF{\NIL}}}}})} 
  Return \retval{next character} from \VAR{stream}.

  \IT{(\FU*{READ-BYTE} \VAR{stream} \OP{\VAR{eof-err}\DF{\T}
      \Op{\VAR{eof-val}\DF{\NIL}}})} 
  Return \retval{next byte} from binary \VAR{stream}.

  \IT{(\FU*{READ-LINE} \OP{\VAR{stream}\DF{\T}
      \OP{\VAR{eof-err}\DF{\T} \Op{\VAR{eof-val}\DF{\NIL}
          \Op{\VAR{recursive}\DF{\NIL}}}}})} 
  Return a \retval{line of text} from \VAR{stream}. Secondary value is \retval{\T} if line was ended by end of file.

  \IT{(\FU*{READ-SEQUENCE} \VAR{sequence} \VAR{stream}
    \OP{\kwd{:start} \VAR{start}\DF{0}}\Op{\kwd{:end}
      \VAR{end}\DF{\NIL}})} 
  Replace elements of \VAR{sequence} between \VAR{start} and \VAR{end}
  with \VAR{stream}. Return index of \VAR{sequence}'s unmodified
  element. 

  \IT{(\FU*{READ-CHAR-NO-HANG}
    \OP{\VAR{stream}\DF{\kwd{*standard-input*}}
      \OP{\VAR{eof-error}\DF{\T} \Op{\VAR{eof-val}\DF{\NIL}
          \Op{\VAR{recursive}\DF{\NIL}}}}})} 
  \retval{Next character} from \VAR{stream} or \retval{\NIL} if none
  is available.

  \IT{(\FU*{PEEK-CHAR} \OP{\VAR{mode}\DF{\NIL}\OP{\VAR{stream}\DF{\kwd{*standard-input*}}
      \OP{\VAR{eof-error}\DF{\T} \Op{\VAR{eof-val}\DF{\NIL}
          \Op{\VAR{recursive}\DF{\NIL}}}}}})}
  Next, or, if \VAR{mode} is \T, next non-whitespace
  \retval{character} from stream without removing it there.

  \IT{(\FU*{READ-FROM-STRING} \VAR{string}\OP{\VAR{eof-error}\DF{\T}
      \Op{\VAR{eof-val}\DF{\NIL}}}\orGOO{%
      \kwd{:start}\kwd{:start}\VAR{ start}\DF{0}\\
      \kwd{:end}\VAR{ end}\DF{\NIL}\\
      \kwd{:preserve-whitespace}\VAR{ bool}\DF{\NIL}}{\}})}
  Return \retval{object} read from string and \retvalii{position} of
  next character.


  \IT{(\FU*{READTABLE-CASE} \VAR{readtable})}
  \retval{Case sensitivity attribute} of \VAR{readtable}. \kwd{setf}able.

  \IT{(\FU*{COPY-READTABLE} \VAR{from-readtable}\DF{\kwd{*readtable*}}
    \VAR{to}\DF{\NIL})}
  \retval{Copy of \VAR{from-readtable}}.

  \IT{(\FU*{SET-SYNTAX-FROM-CHAR} \VAR{to-char} \VAR{from-char}
    \OP{\VAR{to-readtable}\DF{\kwd{*readtable*}}
      \Op{\VAR{from-readtable}\DF{standard readtable}}})}
  Copy syntax of \VAR{from-char} to \VAR{to-readtable}. Return \retval{\T}.

  \IT{(\FU*{CLEAR-INPUT}
    \Op{\VAR{stream}}\DF{\kwd{*standard-input*}})}
  Clear input from \VAR{stream}, return \retval{\NIL}.

  \IT{(\xorGOO{\FU*{CLEAR-OUTPUT}\\
      \FU*{FORCE-OUTPUT}\\
      \FU*{FINISH-OUTPUT}}{\}}%
    \Op{\VAR{stream}\DF{\kwd{standard-output}}})}
  End output to \VAR{stream} and return \retval{\NIL} immediately,
  after initiating flushing of buffers, or after flushing of buffers,
  resp. 

  \IT{\V*{*read-base*}\DF{10}} 
  Radix for reading \kwd{integer}s and \kwd{ratio}s.

  \IT{\V*{*read-default-float-format*}\DF{\kwd{single-float}}}
  Floating point format to use when not indicated in the number read. 

  \IT{\V*{*readtable*}}
  Current readtable.

  \IT{(\FU*{UNREAD-CHAR} \VAR{character}
    \Op{\VAR{stream}\DF{\kwd{*standard-input*}}})} 
  Put last read \VAR{character} back into \VAR{stream}, return
  \retval{\NIL}. 

  \IT{(\FU*{SET-MACRO-CHARACTER} \VAR{char} \VAR{function}
    \OP{\VAR{non-term-p}\DF{\NIL}
      \Op{\VAR{rt}\DF{\kwd{*readtable*}}}})} 
  Make \VAR{char} a macro character associated with
  \VAR{function}. Return \retval{\T}.

  \IT{(\FU*{GET-MACRO-CHARACTER} \VAR{char}
    \Op{\VAR{rt}\DF{\kwd{*readtable*}}})}
  \retval{Reader macro function} associated with \VAR{char}, and
  \retvalii{\T} if \VAR{char} is a non-terminating macro character.

  \IT{(\FU*{MAKE-DISPATCH-MACRO-CHARACTER} \VAR{char}
    \OP{\VAR{non-term-p}\DF{\NIL}
      \Op{\VAR{rt}\DF{\kwd{*readtable*}}}})}
  Make \VAR{char} a dispatching macro character. Return \retval{\T}.

  \IT{(\FU*{SET-DISPATCH-MACRO-CHARACTER} \VAR{char} \VAR{sub-char} \VAR{function}
    \Op{\VAR{rt}\DF{\kwd{*readtable*}}})}
  Make \VAR{function} a dispatch function of \VAR{char} followed by
  \VAR{sub-char}. Return \retval{\T}.

  \IT{(\FU*{GET-DISPATCH-MACRO-CHARACTER} \VAR{char} \VAR{sub-char}
    \Op{\VAR{rt}\DF{\kwd{*readtable*}}})}
  \VAR{Dispatch function} associated with \VAR{char} followed by \VAR{sub-char}.

\end{LIST}

%%%%%%%%%%%%%%%%%%%%%%%%%%%%%%%%%%%%%%%%%%%%%%%%%%
\subsection{Macro Characters}
%%%%%%%%%%%%%%%%%%%%%%%%%%%%%%%%%%%%%%%%%%%%%%%%%%

\begin{LIST}{1cm}

  \IT{\KWD*{(}}
  Initiate reading of a list.

  \IT{\KWD*{"}}
  Begin and end of a string.

  \IT{\KWD*{'}\VAR{foo}} Quote \VAR{foo}.

  \IT{\KWD*{`}\VAR{foo} \KWD*{,}\VAR{bar} \KWD*{,@}\VAR{baz} \KWD*{,.}\VAR{quux} \VAR{bing}}
  Quote \VAR{foo} and \VAR{bing}; but neither \VAR{bar}, nor the lists
  \VAR{baz} and \VAR{quux} which are spliced into their
  elements. \VAR{quux} is possibly modified in the process.

  \IT{\KWD*{\#b}, \KWD*{\#o}, \KWD*{\#x}, \KWD{\#}\VAR{n}\KWD{R}} Binary, octal, or hexadecimal
  number, or number of base \VAR{n}.
  \index{\#R}

  \IT{\KWD*{\#C(}\VAR{a b}\kwd{)}} Complex number \VAR{a}$+$\VAR{b}i.

  \IT{\KWD{\#}\Op{\VAR{n}}\kwd{*}\OPn{\VAR{b}}} 
  \index{\#*}%
  Bit vector of some, or \VAR{n},
  \VAR{b}s filled with last \VAR{b} if necessary.

  \IT{\KWD*{\#S(}\VAR{type} \Goos{\VAR{slot} \VAR{value}}\kwd{)}} 
  Structure of \VAR{type}.

  \IT{\KWD{\#}\VAR{n}\KWD{A}\OPn{\VAR{foo}}} 
  \index{\#A}%
  \VAR{n}-dimensional array

  \IT{\KWD{\#}\Op{\VAR{n}}\kwd{(}\OPn{\VAR{foo}}\kwd{)}} 
  \index{\#(}%
  Vector of some, or \VAR{n},
  \VAR{foo}s filled with last \VAR{foo} if necessary.

  \IT{\KWD*{\#:}\VAR{foo}} Uninterned Symbol \VAR{foo}.

  \IT{\KWD*{\#.}\VAR{form}} Read-time value of \VAR{form}.

  \IT{\KWD{\#}\VAR{label}\kwd{=} \VAR{foo}} Give value of \VAR{foo} a \VAR{label}.
  \index{\#=}%

  \IT{\KWD{\#}\VAR{label}\kwd{\#}} \retval{Object labelled \VAR{label}}.
  \index{\#\#}%

  \IT{\KWD{\#\boldmath{$\backslash$}}\VAR{c}} Character \VAR{c}.
  \index{\#@\#$\backslash$}

  \IT{\KWD*{\#'}\VAR{foo}} Return \retval{function named \VAR{foo}}.

  \IT{\KWD*{\#P}\VAR{foo}} Value of \retval{(\kwd{parse-namestring '}\VAR{foo})}.

  \IT{\KWD*{;}\OPn{\VAR{comment}}} Rest of line is a comment.

  \IT{\KWD{\#\boldmath$|$}\OPn{\VAR{comment}}\KWD{\boldmath$|$\#}} Comment

  \IT{\KWD{\#\boldmath$<$}}
  \index{\#<@\#\boldmath$<$}%
  Have the reader signal an error.

  \IT{\KWD*{\#+}\VAR{test} \VAR{form}}
  Means \VAR{form} if \VAR{test} returns \T, means whitespace otherwise.

  \IT{\KWD*{\#--}\VAR{test} \VAR{form}}
  Means \VAR{form} if \VAR{test} returns \NIL, means whitespace otherwise.

  \IT{\KWD*{:}\VAR{foo}} Keyword, evaluates to \retval{:foo}

  \IT{\V*{*read-eval*}\DF{\T}}
  If \NIL, \kwd{\#.}  \kwd{reader-error} is signalled.

  \IT{\V*{*read-suppress*}\DF{\NIL}}
  Used by \kwd{\#+} and \kwd{\#--}.

\end{LIST}


%%%%%%%%%%%%%%%%%%%%%%%%%%%%%%%%%%%%%%%%%%%%%%%%%%
\subsection{Printer} 
%%%%%%%%%%%%%%%%%%%%%%%%%%%%%%%%%%%%%%%%%%%%%%%%%%


\begin{LIST}{1cm}

  \IT{(\xorGOO{\FU*{PRIN1}\\\FU*{PRINT}\\\FU*{PPRINT}\\\FU*{PRINC}}{\}} \VAR{foo} \Op{stream}\DF{\T})}
  Print \retval{\VAR{foo}} to \VAR{stream} \FU{read}ably,
  \FU{read}ably after a newline and followed by a space,
  \FU{read}ably after a newline, or human-readably without any extra characters, resp.

  \IT{(\FU*{WRITE-CHAR} \VAR{char} \Op{stream}\DF{\T}))}
  Output \retval{\VAR{char}} to \VAR{stream}.

  \IT{(\xorGOO{\FU*{WRITE-STRING}\\\FU*{WRITE-LINE}}{\}} \VAR{string}
    \VAR{stream} 
    \orGOO{\kwd{:start} \VAR{ start}\DF{0}\\\kwd{:end} \VAR{ end}\DF{\NIL}}{\}})}
  Write \retval{part of \VAR{string} between \VAR{start} and \VAR{end}} to \VAR{stream}, without/with a trailing newline, resp.

  \IT{(\FU*{TERPRI} \Op{\VAR{stream}\DF{\T}})}
  Output a newline to \VAR{stream} and return \NIL.

  \IT{(\FU*{FRESH-LINE}) \Op{\VAR{stream}\DF{\T}}}
  Output a newline to \VAR{stream} and return \T\ unless \VAR{stream} is already at the start of a line.

  \IT{(\FU*{PRINT-OBJECT} \VAR{object} \VAR{stream})} \retval{\VAR{object}}

  \IT{(\xorGOO{\FU*{WRITE}\\
      \FU*{WRITE-TO-STRING}}{\}}\VAR{foo}\orGOO{%
      \kwd{:array} \VAR{ bool}\\
      \kwd{:base} \VAR{ radix}\\
      \kwd{:case} \xorGOO{\kwd{:upcase}\\\kwd{:downcase}\\\kwd{:capitalize}}{.}\\
      \kwd{:circle} \VAR{ bool}\\
      \kwd{:escape} \VAR{ bool}\\
      \kwd{:gensym} \VAR{ bool}\\
      \kwd{:length} \VAR{ int}\\
      \kwd{:level} \VAR{ int}\\
      \kwd{:lines} \VAR{ int}\\
      \kwd{:miser-width} \VAR{ int}\\
      \kwd{:pprint-dispatch} \VAR{ d-tbl}\\
      \kwd{:pretty} \VAR{ bool}\\
      \kwd{:radix} \VAR{ bool}\\
      \kwd{:readably} \VAR{ bool}\\
      \kwd{:right-margin} \VAR{ int}\\
      \kwd{:stream} \VAR{ stream}\DF{\T}%
    }{\}})} 
  Print and return \retval{\VAR{foo}}, or print into
  \retval{string}, resp., after dynamically setting printer variables
  corresponding to keyword parameters (\kwd{:}\VAR{bar} becoming
  \kwd{*print-}\VAR{bar}\kwd{*}). (\kwd{:stream} keyword for \FU{write} only.)

  \IT{(\FU*{WRITE-BYTE} \VAR{byte} \VAR{stream})} Write one \retval{\VAR{byte}} to \VAR{stream}.

  \IT{(\FU*{WRITE-SEQUENCE} \VAR{sequence} \VAR{stream}\orGOO{\kwd{:start}\VAR{ start}\DF{0}\\
      \kwd{:end} \VAR{ end}\DF{\NIL}}{\}})} Write elements of \retval{\VAR{sequence}} to \VAR{stream}.

  \IT{\arrGOO{(\FU*{PRIN1-TO-STRING} \VAR{ foo})\\
      (\FU*{PRINC-TO-STRING} \VAR{ foo})}{\}}}
  Print \VAR{foo} to \retval{\VAR{string}} \FU{read}ably or
  human-readably, resp.

  \IT{(\FU*{PPRINT-DISPATCH} \VAR{foo}
    \Op{\VAR{table}\DF{\kwd{*print-pprint-dispatch*}}})}
  Highest priority \retval{\VAR{function}} associated with type of \VAR{foo}.

  \IT{(\xorGOO{\FU*{PPRINT-FILL}\\
      \FU*{PPRINT-LINEAR}}{\}}%
    \VAR{stream} \VAR{foo} \OP{\VAR{parenthesis}\DF{\T} \Op{\VAR{noop}}})}
  Print \VAR{foo} to \VAR{stream}. If \VAR{foo} is a list, print as
  many elements per line as possible; or print either all elements on
  one line or each on its own line. Return \retval{\NIL}.

  \IT{(\FU*{PPRINT-TABULAR} \VAR{stream} \VAR{foo}
    \OP{\VAR{parenthesis}\DF{\T} \Op{\VAR{noop}
        \Op{\VAR{n}\DF{16}}}})} 
  Print \VAR{foo} to \VAR{stream}. If \VAR{foo} is a list, print
  elements in a table with column width \VAR{n} ems.  Return \retval{\NIL}.
  
  \IT{(\SF*{PPRINT-LOGICAL-BLOCK} (\VAR{stream} \VAR{list}
    \orGOO{\xorGOO{\kwd{:prefix} \VAR{ prefix}\\
        \kwd{:per-line-prefix} \VAR{ per-line-prefix}}{\}}\\
      \kwd{:suffix} \VAR{ suffix}\DF{\LIT{""}}}{\}})
    \OPn{(\kwd{declare} \OPn{\VAR{decl}})} \OPn{\VAR{form}})}
  Evaluate \VAR{form}s, which should print \VAR{list}, with
  \VAR{stream} locally bound to a pretty 
  printing stream which outputs to the original \VAR{stream}.

  \begin{LIST}{.5cm}

    \IT{(\SF*{PPRINT-POP})}
    Take \retval{next element} off \VAR{list}. If there is no remaining
    list in \VAR{list}, or \V{*print-length*} or \V{*print-circle*} indicate
    printing should end, send element together with an appropriate to
    \VAR{stream}. 

    \IT{(\FU*{PPRINT-TAB} \Goo{\kwd{:line}\XOR\kwd{:line-relative}%
        \XOR\kwd{:section}\XOR\kwd{:section-relative}} \VAR{c}
      \VAR{i} \Op{\VAR{stream}})}
    Move cursor forward to column number $c + ki$, $k \geq 0$ being as small
    as possible.

    \IT{(\kwd*{PPRINT-EXIT-IF-LIST-EXHAUSTED})}
    If \VAR{list} is empty, terminate logical block. Return
    \retval{\NIL} otherwise.

  \end{LIST}


  \IT{(\FU*{PRINT-NOT-READABLE-OBJECT} \VAR{condition})}
  The \retval{Object} unprintable under \VAR{condition}.

  \IT{(\FU*{COPY-PPRINT-DISPATCH}\Op{\VAR{table}\DF{*print-pprint-dispatch*}})}
  Return \retval{copy of \VAR{table}}.

  \IT{(\FU*{SET-PPRINT-DISPATCH} \VAR{type} \VAR{function}
    \OP{\VAR{priority}\DF{0}
      \Op{\VAR{table}\DF{*print-pprint-dispatch*}}})}
  Install entry comprising \VAR{function} and \VAR{priority} as
  \VAR{type} into \VAR{table}. Return \retval{\NIL}.

  \IT{(\GFU*{DOCUMENTATION} \kwd{'}\VAR{foo}
    \Goo{\kwd{variable}\XOR\kwd{function}\XOR
      \kwd*{compiler-macro}\XOR
      \kwd{method-combination}\XOR
      \T\XOR\kwd{structure}
      \XOR\kwd{type}\XOR\kwd{setf}\XOR\T})}
  Return \retval{documentation string} of \VAR{foo} of given type. \kwd{setf}able.


  \IT{(\kwd*{WITH-COMPILATION-UNIT} (\Op{\kwd{:override}
      \VAR{bool}\DF{\NIL}}) \OPn{\VAR{form}})}
  Return \retval{values of \VAR{form}s}. Warnings deferred by the compiler until end
  of compilation are deferred until the end of evaluation of \VAR{form}s.
  

  \IT{(\SF*{WITH-STANDARD-IO-SYNTAX} \OPn{\VAR{form}})}
  Evaluate \VAR{form}s with standard behaviour of reader and
  printer. Return \retval{values of \VAR{form}s}.

  \IT{\arrGOO{\V*{*standard-input*}\\
      \V*{*standard-output*}\\
      \V*{*error-output*}}{\}}}
  Standard input stream, standard output stream, 
  or standard error output stream, resp.


  \IT{\arrGOO{\V*{*debug-io*}\\
      \V*{*query-io*}}{\}}}
  Streams for debugging and user interaction.

  \IT{\V*{*print-escape*}\DF{\T}} 
  If \NIL, do not print escape characters and package prefixes.

  \IT{\V*{*print-radix*}\DF{\NIL}} 
  If \T, print rationals with a radix indicator.

  \IT{\V*{*print-base*}\DF{10}}
  Radix for printing rationals, between 2 and 36.

  \IT{\V*{*print-circle*}\DF{\NIL}}
  If \T, avoid indefinite recursion while printing circular
  structure. 

  \IT{\V*{*print-pretty*}}
  If \T, print pretty.

  \IT{\V*{*print-level*}\DF{\NIL}}
  If integer, restrict printing of nested object to that depth.

  \IT{\V*{*print-length*}\DF{\NIL}}
  If integer, restrict printing of objects to that number per level.

  \IT{\V*{*print-case*}\DF{\kwd{:upcase}}} 
  Print symbol names all uppercase (\kwd{:upcase}), all lowercase
  (\kwd{:downcase}), capitalized (\kwd{:capitalize}).

  \IT{\V*{*print-gensym*}\DF{\T}} 
  If \T, print \kwd{:\#} before uninterned symbols.

  \IT{\V*{*print-array*}}
  If \T, print array \FU{read}ably.

  \IT{\V*{*print-readably*}\DF{\NIL}}
  If \T, print \FU{read}ably.

  \IT{\V*{*print-lines*}\DF{\NIL}}
  Number to the number of lines to pretty-print.

  \IT{\V*{*print-miser-width*}}

  \IT{\V*{*print-pprint-dispatch*}}
  Current pretty print dispatch table.

  \IT{\V*{*print-right-margin*}\DF{\NIL}}
  Width in ems for the right margin while pretty-printing..

\end{LIST}



%%%%%%%%%%%%%%%%%%%%%%%%%%%%%%%%%%%%%
\subsection{Format}
%%%%%%%%%%%%%%%%%%%%%%%%%%%%%%%%%%%%%
\label{section:Format}

\begin{LIST}{1cm}

  \IT{(\SF*{FORMATTER} \VAR{control})}
  Return \retval{function} applying \FU{format} to
  \kwd{*standard-output*}, \VAR{control}, and a \kwd{\&rest} argument
  returning \NIL\ or any excess arguments. 

  \IT{(\FU*{FORMAT} \Goo{\T\XOR\NIL\XOR\VAR{out-string}\XOR\VAR{out-stream}}
    \VAR{control} \OPn{arg})} 
  Output string \VAR{control} which may
  contain \kwd{\TLD}-directives possibly taking some
  \VAR{args}.
  Output to \VAR{out-string}, \VAR{out-stream}, or, if first
  argument is \T, to \V{*standard-output*}. Return \retval{\NIL}. If
  first argument is \NIL, return \retval{formatted output}. 

  \begin{LIST}{.5cm}

    \IT{\Goo{\KWD{\TLD C}\XOR\KWD{\TLD :C}\XOR\KWD{\TLD @:C}}} 
    Print, spell out, or tell how to type, resp., argument as
    (possibly non-printing) character.

    \IT{\Goo{\KWD{\TLD\%}\XOR\KWD{\TLD}\VAR{n}\KWD{\%}}}
    Print one or \VAR{n} newline(s).

    \IT{\Goo{\KWD{\TLD\&}\XOR\KWD{\TLD}\VAR{n}\KWD{\&}}} 
    Print one or \VAR{n} newline(s) unless output stream is at the
    beginning of a line.

    \IT{\Goo{\KWD{\TLD\boldmath$|$}\XOR\KWD{\TLD}\VAR{n}\KWD{\boldmath$|$}}}
    Print one or \VAR{n} page separator(s).

    \IT{\Goo{\KWD{\TLD\TLD}\XOR\KWD{\TLD}\VAR{n}\KWD{\TLD}}}
    Print one or \VAR{n} tilde(s).

    \IT{\KWD{\TLD}\Op{\VAR{radix}\DF{\LIT{10}} \Op{\KWD{,}\VAR{width}
          \Op{\KWD{,}\VAR{pad-char}\DF{\KWD{\#$\backslash$Space}}
            \Op{\KWD{,}\VAR{comma-char}\DF{\KWD{,}}
              \Op{\KWD{,}\VAR{comma-interval}\DF{\LIT{3}}}}}}} \KWD{\Op{:}\Op{@}R}}
    Print argument as number, with \KWD{:}, group digits 
    \VAR{comma-interval} each, with \kwd{@}, always prepend a sign.

    \IT{\Goo{\KWD{\TLD R}\XOR\KWD{\TLD :R}\XOR\KWD{\TLD @R}\XOR\KWD{\TLD @:R}}} 
    Take argument as number and print it as English cardinal number,
    as English ordinal number, as Roman numeral, or as old Roman
    numeral, resp.

    \IT{\KWD{\TLD}\Op{\VAR{width}
        \Op{\KWD{,}\VAR{pad-char}\DF{\KWD{\#$\backslash$Space}}
          \Op{\KWD{,}\VAR{comma-char}\DF{\KWD{,}}
            \Op{\KWD{,}\VAR{comma-interval}\DF{\LIT{3}}}}}} \KWD{\Op{:}\Op{@}\Goo{D\XOR
          B\XOR O\XOR X}}}
    Print integer argument as number (decimal, binary, octal, or
    hexadecimal, resp.) with \kwd{:} group digits 
    \VAR{comma-interval} each, with \kwd{@} always prepend a sign.

    \IT{\KWD{\TLD}\Op{\VAR{width}\Op{\KWD{,}\VAR{dec-digits} \Op{\KWD{,}\VAR{shift}\DF{\LIT{0}}
            \Op{\KWD{,}\VAR{overflow-char}
              \Op{\KWD{,}\VAR{pad-char}\DF{\LIT{\#$\backslash$Space}}}}}}} \KWD{\Op{@}F}}
    Print argument as floating-point number. With \kwd{@}, always prepend a sign.

    \IT{\KWD{\TLD}\Op{\VAR{width}\Op{\KWD{,}\VAR{int-digits} \Op{\KWD{,}\VAR{exp-digits}
            \Op{\KWD{,}\VAR{scale-factor}\DF{\LIT{1}} \Op{\KWD{,}\VAR{overflow-char}
                \Op{\KWD{,}\VAR{pad-char}\DF{\LIT{\#$\backslash$Space}}
                  \Op{\KWD{,}\VAR{exp-char}}}}}}}} \KWD{\Op{@}\Goo{E\XOR  G}}}
    Print argument as floating-point number with \VAR{int-digits}
    before decimal point and \VAR{exp-digits} in the
    exponent. With \KWD{\TLD G}, chose either \KWD{\TLD E} or \KWD{\TLD F}. With \KWD{@}, always prepend a sign.

    \IT{\KWD{\TLD}\Op{\VAR{min-col}\DF{\LIT{0}}\Op{\KWD{,}\VAR{col-inc}\DF{\LIT{1}}
          \Op{\KWD{,}\VAR{min-pad}\DF{\LIT{0}}
            \Op{\KWD{,}\VAR{pad-char}\DF{\LIT{\#$\backslash$Space}}}}}} \KWD{\Op{:}\Op{@}\Goo{A\XOR S}}}
    Print argument of any type for consumption by humans/by the
    reader, resp. With \kwd{@}, add \VAR{pad-char}s on the left
    rather than on the right. With \kwd{:} print \NIL\ as \LIT{()} rather
    than \LIT{nil}.

    \IT{\KWD{\TLD\Op{:}\Op{@}W}}
    Print obeying every printer control variable. \kwd{:} makes it
    pretty-print. With \kwd{@}, prints without limits on length or depth.

    \IT{\Goo{\KWD{\TLD}\_\XOR\KWD{\TLD:}\_\XOR\KWD{\TLD@}\_\XOR\KWD{\TLD}:@\_}}
    Print newline like \kwd{pprint-newline} with argument
    \kwd{:linear}, \kwd{:fill}, \kwd{:miser}, or \kwd{:mandatory}, resp.

    \IT{\KWD{\TLD}\Op{\VAR{min-col}\DF{\LIT{0}}
        \Op{\KWD{,}\VAR{col-inc}\DF{\LIT{1}}
          \Op{\KWD{,}\VAR{min-pad}\DF{\LIT{0}}
            \Op{\KWD{,}\VAR{pad-char}\DF{\LIT{\#$\backslash$Space}}}}}} \KWD{\Op{:}\Op{@}\boldmath{$<$}}
      \Op{\VAR{text}\KWD{\TLD}\VAR{a},\VAR{b}\kwd{:;}} \Goos{\VAR{text}\KWD{\TLD;}} \KWD{\TLD\boldmath{$>$}}}
    Justify text produced by \VAR{text}s in a field of at least
    \VAR{min-col} columns.

    \IT{\Goo{\KWD{\TLD}\Op{\VAR{n}\DF{\LIT{0}}}\kwd{I}\XOR\KWD{\TLD}\Op{\VAR{n}\DF{\LIT{0}}}\kwd{:I}}}
    Same as (\kwd{pprint-indent} \kwd{:block} \VAR{n}) or
    (\kwd{pprint-indent} \kwd{:current} \VAR{n}), resp.

    \IT{\KWD{\TLD\Op{:}\Op{@}/}\VAR{foo}\KWD{/}}
    Call function \VAR{foo} with arguments stream, format-argument,
    bool-colon, and bool-at for printing format-argument.

    \IT{\KWD{\TLD}\Op{\VAR{c}\Op{\KWD{,}\VAR{i}}}\Op{\KWD{:}}\Op{\KWD{@}}\KWD{T}}
    Move cursor forward to column number $c + ki$, $k \geq 0$ being as small
    as possible.

    \IT{\KWD{\TLD}\Op{\VAR{n}}\DF{\LIT{0}}\Op{\KWD{:}}\Op{\KWD{@}}\KWD{*}}
    Jump \VAR{n} arguments forward, or, with \kwd{:}, backward. With
    \kwd{@}, jump to argument \VAR{n}.

    \IT{\KWD{\TLD}\Op{\VAR{i}}\Op{\KWD{:}}\Op{\KWD{@}}\KWD{[}\Goos{\VAR{text}\KWD{\TLD;}}\KWD{\TLD]}}
    The \VAR{text}s are clauses the argumenth (or the \VAR{i}th if given) of wich is chosen. With
    \kwd{:}, the boolean argument takes first \VAR{str} for \NIL\ and
    second for \T.

    \IT{\KWD{\TLD}\Op{\VAR{limit}}\Op{\KWD{{:}}\Op{\kwd{@}}}%
      \KWD{\boldmath{$\{$}}\VAR{text}\KWD{\TLD\boldmath{$\}$}}}
    \VAR{text} is used repeatedly, up to \VAR{limit}, as control string for every element
    of argument, which should be a list. With \kwd{@}, take all
    remaining arguments.
    \kwd{:}, argument should be a list of sublists. At each iteration
    step the elements of a sublist can be used by \VAR{str}.

    \IT{\KWD{\TLD\Op{:}\Op{@}?}}
    Process two arguments as format string and argument list. With
    \kwd{@}, take one argument as format string and use then the
    original argument list

    \IT{\Goo{%
        \KWD{\TLD(}\VAR{text}\KWD{\TLD)}\XOR
        \KWD{\TLD:(}\VAR{text}\KWD{\TLD)}\XOR
        \KWD{\TLD @(}\VAR{text}\KWD{\TLD)}\XOR
        \KWD{\TLD:@(}\VAR{text}\KWD{\TLD)}}}
    Convert to lowercase, convert first letter of each word to
    uppercase, capitalize first word and convert
    the rest to lowercase, or convert to uppercase, resp.

    \IT{\Goo{\KWD{\TLD P}\XOR\KWD{\TLD:P}
        \XOR\KWD{\TLD @P}\XOR\KWD{\TLD:@P}}}
    If argument \kwd{eql} 1 print nothing, otherwise print \LIT{s};
    do the same after backing up 1 argument; if argument \kwd{eql} 1
    print \LIT{y}, otherwise print \LIT{ies}, resp.

    \IT{\KWD{\TLD\^{}}}
    Leave immediately \kwd{\TLD\boldmath{$<$}} or \kwd{\TLD\boldmath{$\{$}}.

    \IT{\KWD{\TLD}\Op{\KWD{@}}}
    (Tilde-newline) Ignore newline and following
    whitespace. With \kwd{@}, ignore only newline.

  \end{LIST}

\end{LIST}


%%%%%%%%%%%%%%%%%%%%%%%%%%%%%%%%%%%%%%%%%%%%%%%%%%
\subsection{Streams} 
%%%%%%%%%%%%%%%%%%%%%%%%%%%%%%%%%%%%%%%%%%%%%%%%%%

\begin{LIST}{1cm}

  \IT{(\FU*{OPEN} \VAR{path}
    \orGOO{\kwd{:direction} \xorGOO{\kwd{:input}\\\kwd{:output}\\\kwd{:io}\\\kwd{:closed}}{.}\\
      \kwd{:element-type} \VAR{ type}\\
      \kwd{:if-exists}
      \xorGOO{\kwd{:new-version}\\\kwd{:rename}\\
        \kwd{:rename-and-delete}\\\kwd{:overwrite}\\
        \kwd{:append}\\\kwd{:supersede}\\\NIL}{.}\\
      \kwd{:if-does-not exist} \xorGOO{\kwd{:error}\\
        \kwd{:create}\\\NIL}{.}}{\}})}
  Open \retval{stream to \VAR{path}}.

  \IT{(\FU*{CLOSE} \VAR{stream} \Op{\kwd{:abort} \VAR{bool}\DF{\NIL}})}
  Close \VAR{stream}. If \kwd{:abort} is \T, delete associated file.

  \IT{(\FU*{MAKE-STRING-OUTPUT-STREAM} \Op{\kwd{:element-type}
      \VAR{type}\DF{\kwd{character}}})} 
  Return a \retval{stream} accepting characters.

  \IT{(\FU*{GET-OUTPUT-STREAM-STRING} \VAR{stream})}
  Clear and return as a \retval{string} characters on \VAR{stream}.

  \IT{(\FU*{MAKE-BROADCAST-STREAM} \OPn{\VAR{stream}})}
  Return \retval{broadcast stream}.

  \IT{(\FU*{MAKE-CONCATENATED-STREAM} \OPn{\VAR{stream}})}
  Return \retval{concatenated} stream.

  \IT{(\SF*{WITH-INPUT-FROM-STRING} (\VAR{foo} \VAR{string} 
    \orGOO{\kwd{:index} \VAR{ index}\\\kwd{:start} \VAR{
        start}\\\kwd{:end} \VAR{ end}}{\}}) \OPn{(\kwd{declare}
      \OPn{\VAR{decl}})} \OPn{\VAR{form}})} 
  Evaluate \VAR{form}s with \VAR{foo} locally bound to input stream
  from \VAR{string}. \VAR{index} contains the reading position in
  \VAR{string} after leaving. Return \retval{value of \VAR{form}s}. 

  \IT{(\SF*{WITH-OUTPUT-TO-STRING}(\VAR{foo} 
    \Op{\VAR{string}}\DF{\NIL} \Op{\kwd{:element-type}
      \VAR{type}\DF{\kwd{character}}}) \OPn{(\kwd{declare}
      \OPn{\VAR{decl}})} \OPn{\VAR{form}})} 
  Evaluate \VAR{form}s with \VAR{foo} locally bound to output
  stream. Append output to \VAR{string} and return \retval{value of
    \VAR{foo}} if \VAR{string} is given. Return \retval{string
    containing output} otherwise. 

  \IT{(\SF*{WITH-OPEN-STREAM} (\VAR{foo} \VAR{stream})
    \OPn{(\kwd{declare} \OPn{\VAR{decl}})} \OPn{\VAR{form}})} 
  Evaluate \VAR{form}s with \VAR{foo} locally bound to
  \VAR{stream}. Return \retval{value of \VAR{form}s}. 

  \IT{(\FU*{BROADCAST-STREAM-STREAMS} \VAR{broadcast-stream})} 
  Return \retval{list of streams} \VAR{broadcast-stream} is
  broadcasting to. 

  \IT{(\FU*{CONCATENATED-STREAM-STREAMS} \VAR{concatenated-stream})} 
  Return \retval{list of streams} \VAR{concatenated-stream} still
  has to read from. 

  \IT{\arrGOO{(\FU*{ECHO-STREAM-INPUT-STREAM}\VAR{ echo-stream})\\
      (\FU*{ECHO-STREAM-OUTPUT-STREAM}\VAR{ echo-stream})}{\}}}
  Return \retval{stream} \VAR{echo-stream} receives from or sends
  to, resp.

  \IT{(\FU*{MAKE-ECHO-STREAM} \VAR{input-stream} \VAR{output-stream})} 
  Place \retval{echo stream} between \VAR{input-stream} and
  \VAR{output-stream}. 

  \IT{(\FU*{MAKE-STRING-INPUT-STREAM} \VAR{string}
    \orGOO{\kwd{:start}\VAR{ start}\\
      \kwd{:end} \VAR{ end}}{\}})}
  Return \retval{stream} supplying the characters from \VAR{string}.

  \IT{(\FU*{MAKE-SYNONYM-STREAM} \VAR{symbol})} Return \retval{synonym
    stream}.

  \IT{(\FU*{MAKE-TWO-WAY-STREAM} \VAR{input-stream}
    \VAR{output-stream})} Return \retval{two-way stream}.

  \IT{(\FU*{STREAM-ELEMENT-TYPE} \VAR{stream})} 
  Return \retval{type} of \VAR{stream} objects.

  \IT{(\FU*{STREAM-EXTERNAL-FORMAT} \VAR{stream})} 
  \retval{External file format designator}.

  \IT{(\FU*{SYNONYM-STREAM-SYMBOL} \VAR{synonym-stream})} 
  Return \retval{symbol} of \VAR{synonym-stream}.

  \IT{\arrGOO{(\FU*{TWO-WAY-STREAM-INPUT-STREAM} \VAR{ two-way-stream})\\
      (\FU*{TWO-WAY-STREAM-OUTPUT-STREAM} \VAR{ two-way-stream})}{\}}}
  Return \retval{stream} \VAR{two-way-stream} receives from or sends
  to, resp.

  \IT{(\FU*{LISTEN} \Op{\VAR{stream}\DF{\kwd{*standard-input*}}})}
  \retval{\T} if is a character in input \VAR{stream}.

  \IT{\V*{*terminal-io*}} Bidirectional stream to user terminal.

\end{LIST}


%%%%%%%%%%%%%%%%%%%%%%%%%%%%%%%%%%%%%%%%%%%%%%%%%%
\subsection{Files} 
%%%%%%%%%%%%%%%%%%%%%%%%%%%%%%%%%%%%%%%%%%%%%%%%%%

\begin{LIST}{1cm}

  \IT{(\FU*{PATHNAME} \VAR{path})}
  Return \retval{pathname}.

  \IT{\V*{*default-pathname-defaults*}}
  Pathname to use if one is needed but none supplied.

  \IT{(\xorGOO{\FU*{PATHNAME-HOST}\\
      \FU*{PATHNAME-DEVICE}\\
      \FU*{PATHNAME-DIRECTORY}\\
      \FU*{PATHNAME-NAME}\\
      \FU*{PATHNAME-TYPE}\\
      \FU*{PATHNAME-VERSION}}{\}}
    \VAR{path}
    \Op{\xorGOO{\kwd{:local}\\\kwd{:common}}{\}}\DF{\kwd{:local}}})}
  Return \retval{pathname component}.

  \IT{(\FU*{MAKE-PATHNAME} 
    \orGOO{\kwd{:host} \VAR{ host}\\
      \kwd{:device} \VAR{ dev}\\
      \kwd{:directory} \VAR{ dir}\\
      \kwd{:name} \VAR{ name}\\
      \kwd{:type} \VAR{ type}\\
      \kwd{:version} \VAR{ ver}\\
      \kwd{:defaults} \VAR{ path}\\
      \kwd{:case \Goo{\kwd{:local}\XOR\kwd{:common}}\DF{\kwd{:local}}}}{\}})} 
  Construct \retval{pathname}.

  \IT{(\FU*{MERGE-PATHNAMES} \VAR{pathname}
    \Op{\VAR{default-pathname}\DF{\V{*default-pathname-defaults*}}}\XOR\Op{\VAR{version}\DF{\kwd{:newest}}})} 
  Return \retval{\VAR{pathname}} after filling in missing parts from defaults.

  \IT{(\FU*{ENOUGH-NAMESTRING} \VAR{path}
    \Op{\VAR{root-path}\DF{\V{*default-pathname-defaults*}}})}
  Return \retval{minimal path string} to sufficiently describe path relative to \VAR{root-path.}

  \IT{\arrGOO{(\FU*{NAMESTRING}\VAR{ path})\\
      (\FU*{FILE-NAMESTRING}\VAR{ path})\\
      (\FU*{DIRECTORY-NAMESTRING}\VAR{ path})\\
      (\FU*{HOST-NAMESTRING}\VAR{ path})}{\}}}
  Return string representing \retval{full pathname}, \retval{name, type, and version},
  \retval{directory name}, or \retval{host name}, resp., of \VAR{path}.

  \IT{(\FU*{PARSE-NAMESTRING} \VAR{foo} \VAR{host} \VAR{default-pathname}\DF{\V{*default-pathname-defaults*}}
    \orGOO{\kwd{:start} \VAR{ start}\DF{\LIT{0}}\\
      \kwd{:end} \VAR{ end}\DF{\NIL}\\
      \kwd{:junk-allowed} \VAR{ bool}\DF{\NIL}}{\}})}
  Return \retval{pathname} and \retval{position} converted from
  string, pathname, or stream \VAR{foo}.

  \IT{(\FU*{PROBE-FILE} \VAR{path})}
  Return \retval{canonical file name} or \retval{\NIL} if file does
  not exist.

  \IT{(\FU*{DELETE-FILE} \VAR{path})}
  Delete file \VAR{path}, return \retval{\T}.

  \IT{(\FU*{RENAME-FILE} \VAR{foo} \VAR{bar})}
  Rename file  \VAR{foo} to \VAR{bar}. Unspecified parts of path
  \VAR{bar} default to those of \VAR{foo}. Return \retval{new
    pathname}, \retval{old file name}, and \retval{new file name}.

  \IT{(\FU*{DIRECTORY} \VAR{path})}
  Return \retval{list of pathnames}.

  \IT{(\FU*{ENSURE-DIRECTORIES-EXIST} \VAR{path} \Op{\kwd{:verbose}
      \VAR{bool}})}
  Create parts of \retval{path} if necessary. Second return value is
  \retval{\T} if something was created.

  \IT{(\FU*{FILE-WRITE-DATE} \VAR{path})} 
  Return \retval{time} at which file was last written.

  \IT{(\FU*{FILE-AUTHOR} \VAR{path})}
  Return \retval{name of file owner}.

  \IT{(\FU*{FILE-LENGTH} \VAR{stream})}
  Return \retval{length of \VAR{stream}}.

  \IT{(\FU*{FILE-POSITION} \VAR{stream} \Op{\VAR{position}})}
  Return \retval{position within stream}, or set it to
  \retval{\VAR{position}} and return \retval{\T} on success. 

  \IT{(\SF*{WITH-OPEN-FILE} (\VAR{stream path}
    \OPn{\VAR{option-form}}) \OPn{(\kwd{declare} \OPn{\VAR{decl}})}
    \OPn{\VAR{form}})} 
  Use \FU*{OPEN} with arguments \VAR{option-form}s to temporarily
  create \VAR{stream}, and return \retval{value of \VAR{form}s}.




  \IT{(\FU*{FILE-STRING-LENGTH} \VAR{stream} \VAR{foo})} 
  \retval{Length} \VAR{foo} would have in \VAR{stream}.

  \IT{(\FU*{FILL-POINTER} \VAR{vector})} Return \retval{fill pointer} of \VAR{vector}. \kwd{setf}able.

  \IT{(\FU*{LOGICAL-PATHNAME} \VAR{path})}
  \retval{Logical name} of \VAR{path}.

  \IT{(\FU*{LOAD-LOGICAL-PATHNAME-TRANSLATIONS} \VAR{host})}
  Load \VAR{host}'s translations. Return \retval{\NIL} if already
  loded, return \retval{\T} if successful.

  \IT{(\FU*{LOGICAL-PATHNAME-TRANSLATIONS} \VAR{host})}
  \retval{\VAR{host}'s list of translations}.

  \IT{(\FU*{PATHNAME} \VAR{path})} 
  \retval{Pathname} of \VAR{path}.

  \IT{(\FU*{TRANSLATE-LOGICAL-PATHNAME} \VAR{path})} 
  Physical \retval{pathname} of \VAR{path}. 

  \IT{(\FU*{TRANSLATE-PATHNAME} \VAR{path-a} \VAR{path-b}
    \VAR{path-c})}
  Translate \VAR{path-a} from wildcard \VAR{path-b} into wildcard
  \VAR{path-c}. Return \retval{new path}.

  \IT{(\FU*{USER-HOMEDIR-PATHNAME} \Op{\VAR{host}})} 
  User's \retval{ home directory}.

  \IT{(\FU*{TRUENAME} \VAR{file})}
  \retval{Canonical name} of \VAR{file}.


\end{LIST}



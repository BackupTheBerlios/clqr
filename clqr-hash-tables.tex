% Copyright (C) 2008 Bert Burgemeister
%
% Permission is granted to copy, distribute and/or modify this
% document under the terms of the GNU Free Documentation License,
% Version 1.2 or any later version published by the Free Software
% Foundation; with no Invariant Sections, no Front-Cover Texts and
% no Back-Cover Texts. For details see file COPYING.
%

%%%%%%%%%%%%%%%%%%%%%%%%%%%%%%%%%%%%%%%%%%%%%%%%%%
\section{Hash Tables} 
%%%%%%%%%%%%%%%%%%%%%%%%%%%%%%%%%%%%%%%%%%%%%%%%%%
Key-value storage similar to hash tables can as well be achieved with
association lists and with property lists; see
pp.\ \pageref{section:Association Lists} and
\pageref{:property_lists}. 


\begin{LIST}{1cm}

  \IT{(\FU*{HASH-TABLE-P} \VAR{foo})}
  Return \retval{\T} if \VAR{foo} is of type \kwd{hash-table}.

  \IT{(\FU*{MAKE-HASH-TABLE} 
    \orGOO{\kwd{:test}\Goo{\kwd{\#'eq}\XOR \kwd{\#'eql}\XOR \kwd{\#'equal}}\DF{\kwd{eql}}\\
      \kwd{:size} \VAR{ size}\\
      \kwd{:rehash-size} \VAR{ rehash-size}\\
      \kwd{:rehash-threshold} \VAR{ rehash-threshold}}{\}})} 
  Make a \retval{hash table}. 

  \IT{(\FU*{HASH-TABLE-TEST} \VAR{hash-table})} \retval{Test function}
  used in \VAR{hash-table}.

  \IT{(\FU*{GETHASH} \VAR{key} \VAR{hash-table} \Op{\VAR{default}})} Get
  \retval{object with key \VAR{key} } if any or \retval{\VAR{default}}
  otherwise. Second return value: \T\ if found, \NIL\ otherwise

  \IT{(\FU*{REMHASH} \VAR{key} \VAR{hash-table})} 
  Remove from \VAR{hash-table} entry with \VAR{key} and return \retval{\T} if it
  existed. Return \retval{\NIL} otherwise. 

  \IT{(\FU*{CLRHASH} \VAR{hash-table})} 
  Empty \retval{\VAR{hash-table}}.


  \IT{(\FU*{HASH-TABLE-COUNT} \VAR{hash-table})} \retval{Number of entries} in \VAR{hash-table}.

  \IT{\arrGOO{(\FU*{HASH-TABLE-SIZE} \VAR{ hash-table})\\
      (\FU*{HASH-TABLE-REHASH-SIZE} \VAR{ hash-table})\\
      (\FU*{HASH-TABLE-REHASH-THRESHOLD} \VAR{ hash-table})}{\}}}
  Current \retval{size}, \retval{rehash-size}, or
  \retval{rehash-threshold}, resp., as used in \FU{make-hash-table}. 

  \IT{(\FU*{MAPHASH} \VAR{function} \VAR{hash-table})} 
  Iterate over \VAR{hash-table} calling \VAR{function} on \VAR{key}
  and \VAR{value}. Return \retval{\NIL}.
  
  \IT{(\SF*{WITH-HASH-TABLE-ITERATOR} \VAR{name} \VAR{hash-table}
    \OPn{(\kwd{declare} \OPn{\VAR{decl}})} \OPn{form})}
  Return \retval{values of \VAR{form}s}. In the body, successive
  invocations of  (\VAR{name}) return: \T\ if an entry is returned;
  its key; its value.

  \IT{(\FU*{SXHASH} \VAR{foo})} \retval{Hash code} of \VAR{foo}.

\end{LIST}



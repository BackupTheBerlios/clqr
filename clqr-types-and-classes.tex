% Copyright (C) 2008, 2009, 2010 Bert Burgemeister
%
% Permission is granted to copy, distribute and/or modify this
% document under the terms of the GNU Free Documentation License,
% Version 1.2 or any later version published by the Free Software
% Foundation; with no Invariant Sections, no Front-Cover Texts and
% no Back-Cover Texts. For details see file COPYING.
%

%%%%%%%%%%%%%%%%%%%%%%%%%%%%%%%%%%%%%%%%%%%%%%%%%%
\section{Types and Classes} 
%%%%%%%%%%%%%%%%%%%%%%%%%%%%%%%%%%%%%%%%%%%%%%%%%%

\begin{figure}
  \begin{center}
    \begin{sideways}
      \includegraphics{clqr-types-and-classes.1}
    \end{sideways}
  \end{center}
  \caption[]{Precedence Order of System Classes
    (\includegraphics{clqr-types-and-classes.3}),
    Classes (\includegraphics{clqr-types-and-classes.4}),
    Types (\includegraphics{clqr-types-and-classes.2}), and
    Condition Types
    (\includegraphics{clqr-types-and-classes.5}).
    \label{data-types}%
    %
    \index{*@\A}%
    \index{T}%
    \index{ATOM}%
    \index{READTABLE}%
    \index{PACKAGE}%
    \index{SYMBOL}%
    \index{KEYWORD}%
    \index{BOOLEAN}%
    \index{RESTART}%
    \index{RANDOM-STATE}%
    \index{HASH-TABLE}%
    \index{STRUCTURE-OBJECT}%
    \index{STANDARD-OBJECT}%
    \index{NULL}%
    \index{CLASS}%
    \index{BUILT-IN-CLASS}%
    \index{STANDARD-CLASS}%
    \index{STRUCTURE-CLASS}%
    \index{METHOD}%
    \index{STANDARD-METHOD}%
    \index{METHOD-COMBINATION}%
    \index{CHARACTER}%
    \index{FUNCTION}%
    \index{COMPILED-FUNCTION}%
    \index{GENERIC-FUNCTION}%
    \index{STANDARD-GENERIC-FUNCTION}%
    \index{PATHNAME}%
    \index{LOGICAL-PATHNAME}%
    \index{NUMBER}%
    \index{COMPLEX}%
    \index{REAL}%
    \index{FLOAT}%
    \index{SHORT-FLOAT}%
    \index{SINGLE-FLOAT}%
    \index{DOUBLE-FLOAT}%
    \index{LONG-FLOAT}%
    \index{RATIONAL}%
    \index{INTEGER}%
    \index{RATIO}%
    \index{SIGNED-BYTE}%
    \index{FIXNUM}%
    \index{BIGNUM}%
    \index{UNSIGNED-BYTE}%
    \index{BIT}%
    \index{LIST}%
    \index{SEQUENCE}%
    \index{CONS}%
    \index{ARRAY}%
    \index{SIMPLE-ARRAY}%
    \index{VECTOR}%
    \index{STRING}%
    \index{SIMPLE-STRING}%
    \index{BASE-STRING}%
    \index{SIMPLE-BASE-STRING}%
    \index{SIMPLE-VECTOR}%
    \index{BIT-VECTOR}%
    \index{SIMPLE-BIT-VECTOR}%
    \index{STREAM}%
    \index{FILE-STREAM}%
    \index{TWO-WAY-STREAM}%
    \index{SYNONYM-STREAM}%
    \index{STRING-STREAM}%
    \index{BROADCAST-STREAM}%
    \index{CONCATENATED-STREAM}%
    \index{ECHO-STREAM}%
    \index{EXTENDED-CHAR}%
    \index{BASE-CHAR}%
    \index{STANDARD-CHAR}%
    \index{CONDITION}%
    \index{SERIOUS-CONDITION}%
    \index{STORAGE-CONDITION}%
    \index{SIMPLE-TYPE-ERROR}%
    \index{TYPE-ERROR}%
    \index{ERROR}%
    \index{PROGRAM-ERROR}%
    \index{CONTROL-ERROR}%
    \index{PACKAGE-ERROR}%
    \index{PRINT-NOT-READABLE}%
    \index{STREAM-ERROR}%
    \index{PARSE-ERROR}%
    \index{CELL-ERROR}%
    \index{FILE-ERROR}%
    \index{ARITHMETIC-ERROR}%
    \index{SIMPLE-CONDITION}%
    \index{WARNING}%
    \index{STYLE-WARNING}%
    \index{SIMPLE-ERROR}%
    \index{SIMPLE-WARNING}%
    \index{UNDEFINED-FUNCTION}%
    \index{UNBOUND-VARIABLE}%
    \index{UNBOUND-SLOT}%
    \index{END-OF-FILE}%
    \index{READER-ERROR}%
    \index{DIVISION-BY-ZERO}%
    \index{FLOATING-POINT-INEXACT}%
    \index{FLOATING-POINT-OVERFLOW}%
    \index{FLOATING-POINT-UNDERFLOW}%
    \index{FLOATING-POINT-INVALID-OPERATION}%
  }
\end{figure}

For any class, there is always a corresponding type of the same name.

\begin{LIST}{1cm}

  \IT{(\FU*{TYPEP} \VAR{foo} \VAR{type} \Op{\VAR{environment}\DF{\NIL}})}
  {
  Return \retval{\T} if \VAR{foo} is of \VAR{type}.
  }

  \IT{(\FU*{SUBTYPEP} \VAR{type-a} \VAR{type-b}
    \Op{\VAR{environment}})}
  {
  Return \retval{\T} if \VAR{type-a} is a recognizable subtype of
  \VAR{type-b}, and \retvalii{\NIL} if the relationship could not be
  determined. 
  }

  \IT{(\SO*{THE} \NEV{\VAR{type}} \VAR{form})}
  {
  Return \retval{values} of \VAR{form} which are declared to be of
  \VAR{type}. 
  }

  \IT{(\FU*{COERCE} \VAR{object} \VAR{type})}
  {
  Coerce \VAR{object} into \VAR{type}.
  }

  \IT{(\MC*{TYPECASE} \VAR{foo} \OPn{(\NEV{\VAR{type}} \PROGN{\VAR{a-form}})} 
    \OP{(\xorGOO{\kwd*{OTHERWISE}\\
        \T}{\}} \PROGN{\VAR{b-form}\DF{\NIL}})})}
  {
  Return \retval{values of the \VAR{a-form}s} whose \VAR{type} is
  \VAR{foo} of. Return \retval{values of \VAR{b-form}s} if no
  \VAR{type} matches.
  }

  \IT{(\xorGOO{\MC*{CTYPECASE}\\
      \MC*{ETYPECASE}}{\}} \VAR{foo} \OPn{(\NEV{\VAR{type}}
      \PROGN{\VAR{form}})})}
  {
  Return \retval{values of the \VAR{form}s} whose \VAR{type} is
  \VAR{foo} of. Signal correctable/non-correctable error, respectively
  if no \VAR{type} matches.
  }

  \IT{(\FU*{TYPE-OF} \VAR{foo})}
  {
  \retval{Type of \VAR{foo}}.
  }

  \IT{(\MC*{CHECK-TYPE} \VAR{place} \VAR{type}
    \Op{\VAR{string}\DF{\Goo{\LIT{a}\XOR\LIT{an}} \VAR{type}}})}
  {
    Signal correctable \kwd{type-error} if \VAR{place} is
    not of \VAR{type}. Return \retval{\NIL}.
  }

  \IT{(\FU*{STREAM-ELEMENT-TYPE} \VAR{stream})}
  {
  Return \retval{type} of \VAR{stream} objects.
  }

  \IT{(\FU*{ARRAY-ELEMENT-TYPE} \VAR{array})}
  {
  Element \retval{type} \VAR{array} can hold.
  }

  \IT{(\FU*{UPGRADED-ARRAY-ELEMENT-TYPE} \VAR{type}
    \Op{\VAR{environment}\DF{\NIL}})}
  {
  \retval{Element type} of most specialized array capable of holding
  elements of \VAR{type}. 
  }

  \IT{(\MC*{DEFTYPE} \VAR{foo} (\OPn{\VAR{macro-$\lambda$}})
    \OPn{(\kwd{declare} \OPn{\NEV{\VAR{decl}}})} \Op{\NEV{\VAR{doc}}} 
    \PROGN{\VAR{form}})}
  {
  Define type \retval{\VAR{foo}} which when referenced as (\VAR{foo}
  \OPn{\NEV{\VAR{arg}}}) applies expanded \VAR{form}s to \VAR{arg}s
  returning the new type. For (\OPn{\VAR{macro-$\lambda$}}) see
  p.~\pageref{section:Macros} but with default value of \kwd{\A}
  instead of \NIL. \VAR{form}s are enclosed in an implicit \SO{block}
  named \VAR{foo}.
  }

  \IT{\arrGOO{(\kwd*{EQL } \VAR{foo})\\
      (\kwd*{MEMBER } \OPn{\VAR{foo}})}{.}}
  {
  Specifier for a type comprising \VAR{foo} or \VAR{foo}s.
  }

  \IT{(\kwd*{SATISFIES} \VAR{predicate})}
  {
  Type specifier for all objects satisfying \VAR{predicate}. 
  }

  \IT{(\kwd*{MOD} \VAR{n})}
  {
  Type specifier for all non-negative integers $<n$.
  }

  \IT{(\kwd*{NOT} \VAR{type})}
  {
  Complement of type.
  }

  \IT{(\kwd*{AND} \OPn{\VAR{type}}\DF{\T})}
  {
  Type specifier for intersection of \VAR{type}s.
  }

  \IT{(\kwd*{OR} \OPn{\VAR{type}}\DF{\NIL})}
  {
  Type specifier for union of \VAR{type}s.
  }

  \IT{(\kwd*{VALUES} \OPn{\VAR{type}} \OP{\OPn{\kwd{\&optional} \VAR{type}}
      \Op{\kwd{\&rest} \VAR{other-args}}})}
  {
  Type specifier for multiple values.
  }

  \IT{\kwd{\A}}
  {
    \index{*@\A}%
    As a type argument (cf.\ Figure \ref{data-types}): no restriction.
  }
  \end{LIST}



%%% Local Variables: 
%%% mode: latex
%%% TeX-master: "clqr"
%%% End: 
